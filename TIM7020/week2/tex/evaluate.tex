\section{Evaluating the Research Utility}

With the issue clearly defined, how useful is the research in addressing the question of how and when to shift data to the cloud in order to develop BDA capabilities for small-to-mid-sized companies?

The overall status and direction of technology is well covered by \textcite{shengTechnology21stCentury2019}. The numerous tables of research categories will help practitioners quickly locate relevant studies and understand the types of questions academics are struggling with. Practitioners will also be well-served to read \textcite{lopez-robles30YearsIntelligence2019} in order to understand the importance of establishing normative terminology while discussing topics related to BDA and Business Intelligence. The authors adequately lay out not only the importance of the topics, but how the various concepts relate to each other.

In order to address the question system development, and provide a case study of how at least some big-data systems can be developed quickly, \textcite{jiModelingImageVideo2019} provides a  case study. However, it has limited utility as it does not provide the necessary insight into how to choose underlying BDA architectures. Still, it clearly shows that developing big-data systems are within the reach of companies that dedicate themselves to the effort.

Companies with specific goals would be well served to references \textcite{rossmannFutureSocialImpact2018} simply to note the importance of evaluating expert advice from multiple domains. The disparity between the views of seasoned experts can vary wildly, and this paper is a useful reminder that practitioners should seek a variety of viewpoints around any complex subject. Further, the paper provides good examples of how and to some extent why assessments of BDA utility differs by domain expertise.

Given the risks associated with BDA investment for small-to-mid-sized companies, \textcite{bozicBusinessIntelligenceAnalytics2019} is an important theoretical work to understand. By proposing and testing a model to show how BDA translates into corporate performance, the authors outline the importance of non-technical considerations. Success in such endeavors are shown to be strongly influenced by elements that are within leadership control. By understanding how the firm needs to function culturally to be successful at BDA, the practitioner will be able to more fairly evaluate if the firm is ready for a significant BDA investment or not. Indeed, other researchers have found that small-to-medium-sized enterprises often neglect assessing their capability to move their IT infrastructure to their detriment \parencite{carcaryAdoptionCloudComputing2014}.

Practitioners also need to understand that BDA capabilities once built are not magic. The old adage of ``garbage in, garbage out'' will still apply. So on top of evaluating the firm's capability to engage in BDA successfully, \textcite{souibguiDataQualityETL2019} provides meaningful insight into the importance of data quality methods and tools. This study is particularly useful for its practical, empirical testing and comparison of four different data quality tools.

\textcite{chengFacilitatingSpeedInternationalization2020} and \textcite{lozadaBigDataAnalytics2019} both outline the value that BDA capabilities can have for the small-to-mid-sized company. These studies look at internationalization and co-innovation respectfully. These are areas where small-to-mid-sized companies, by virtue of their ability to be agile, can utilize BDA and cloud capabilities to outmaneuver larger, slower to change firms. At a practical level, reaffirming this advantage that firms in this size bracket posses can help in evaluating when the right time to move is.

\section{Conclusion}

A key capability for businesses is becoming the ability to have a cost-effective, scaling, resilient, reliable BDA tool-set. This reality is recognized by business leadership. Business leaders in small-to-mid-sized companies have a problem with deciding when to shift their IT environment to a cloud-based, BDA capable environment. This shift can offer firms many competitive advantages, but comes with numerous risks. For small-to-mid-sized companies, the causes of the greatest risks are predominately tied to a single underlying cause: their limited ability to attract adequately skilled talent in this domain. Other causes include the complexity of legacy systems and the multi-vendor reality of the cloud marketplace. However, this additional causes are made even worse for companies in this bracket because of the talent shortage.

The research examined offers numerous practical insights for those trying to make this decision. However, none of these papers on their own or collectively provide any real strategy for addressing the primary reason that these companies are struggling with this question: talent. With a global shortage of adequately trained BDA and cloud engineers, and inadequate abilities to compete for talent based on pay, these companies face a very limited capacity to address the problem directly. Industry watchers such as Gartner see this shortage continuing for many years, and thus, the problem will likely remain unresolved for some time.

Still, these companies must remain aware of the numerous advantages such capabilities give them. And while risky, for companies that have the proper culture and leadership capacity, pursuing a cloud-based BDA architecture can be greatly beneficial.
