\section{Introduction}

Cloud computing was one of the fastest growing areas of IT in 2020 \parencite{gartnerTrendsImpactingCloud2020}. It can provide for very cost-effective solutions and allows for numerous advantages in terms of performance, availability, scalability, reliability, to name a few \parencite{ibmBenefitsCloudComputing2020}.  Large-scale integrators and consulting firms are investing billions to support cloud services for clients and are making cloud computing central to their core strategies \parencite{accentureAccentureCloudFirst2020}. For its many benefits, choosing cloud computing comes with risks for businesses. These include the costs of moving to cloud, disruptions during the move, training needs and costs, compliance requirements, latency, and incompatibilities with current system architectures \parencite{cookCloudMigrationRisks2019}.

For all of these risks, however, cloud computing is the most cost-effective, and easiest means for companies to engage in ``massive-scale and complex computing,'' it is the natural platform upon which to build big data analytics (BDA) tools \parencite{hashemRiseBigData2015}. Companies with aspirations to leverage BDA capabilities in customer management, supply-chain management, and business information tools are well advised to pursue a cloud strategy \parencite{taliaCloudsScalableBig2013}.

Small-to-mid-sized companies are facing a serious strategic problem: when and how to develop BDA capabilities given the associated up-front costs and risks while balancing the long-term benefits \parencite{ajimokoConsiderationsAdoptionCloud2018}. This is a problem that nearly all companies must contend with if they wish to utilize the benefits that BDA provides. The simple reality is that data driven companies are more productive and more profitable than their competitors \parencite{alsghaierImportanceBigData2017}. Yet the risks to small-to-mid-sized companies, with inherently limited IT budgets and less access to highly talented IT staff, are significant. Researchers have noted that there is little understanding of how BDA investment helps some companies become a successful data driven company but not others \parencite{morenoBusinessIntelligenceAnalytics2020}.
