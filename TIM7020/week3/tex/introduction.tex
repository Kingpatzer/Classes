\section{Analysis of Current State}

BlackBean is a financially successful tea and dessert retailer. BlackBean operates out of Hanoi, Vietnam, and is an international  business with a global supply chain. It has 60 locations, with 40 of those locations spread throughout Asia, the remaining 20 being located domestically. BlackBean prepares its deserts in regionally located centers that are strategically located to serve multiple retail locations.

BlackBean is active on social media and promotes itself as a conscientious choice for discerning consumers. Like similar companies such as DavidsTea in Quebec and Alter Eco in San Francisco, BlackBean is aware that consumers are becoming more sophisticated and concerned with issues around supply chain equity and corporate practices. BlackBean seeks to use fair trade, sustainable components in its products. Because of the importance of freshness in many of its ingredients, perishable goods are sourced locally for its regional operations centers. For non-perishable items, they are sourced on the global market and shipped to Hanoi from whence they are distributed to the regional centers. These non-perishable items are sourced first for quality and ethical, sustainable production methods and only then for cost. BlackBean is aware that consumers will pay a premium for luxury items, such as desserts and high-end teas, which are ethical, environmentally friendly, and support fair trade practices.

BlackBean's social media campaigns are coordinated with local stores creating promotions and posts to highlight what is happening locally within the company. Corporate marketing provides talking points that can be used and crafted with relative discretion by the regional managers. Each retail location collects point-of-sale (POS) data and collates the information about sales, promotion results, and customer reactions. This information is returned to corporate headquarters, but the lack of standard reporting formats, promotional data, and the free form nature of some of the data makes the reporting of limited organizational value.

BlackBean's IT staff consists of a small number of individuals who are focused on running their in-house accounting system, the corporate supply chain ordering system, maintaining the marketing database for the marketing team, and providing call-center assistance for desktop computing needs for the head office, regional operations centers, and the individual retail stores. POS updates typically happen quarterly, and BlackBean pays the POS vendor to update these systems. The in-house team are well-versed in desktop support. There are two individuals who are skilled at operating traditional SQL databases, writing queries, and producing reports. The team reports to a manager who sits under the CFO, as the IT team grew out of a need to build and maintain the accounting and POS systems starting in the mid-1990s.

BlackBean's critical strengths are its market presence, market reputation both for quality and ethical operations, and its customer loyalty. BlackBean's critical weaknesses are the lack of modernized IT infrastructure, the lack of operational controls across the environment, the lack of fully integrated POS systems, and a somewhat fragile and hard to manage supply chain. BlackBean's lack of experienced IT staff is a weakness for IT strategic initiatives, but is not a structural problem. However, the fact that IT is a sub-department of the Chief Financial Officer and not either its own department or under Operations is a structural issue that likely will present challenges as IT becomes more deeply involved in supply chain, operational support, and marketing functions.



\section{Functional Needs and Constraints}

BlackBean tea company has a set of problems it requires a strategy to overcome. First, it has multiple Information Technology platform needs. It requires customer relationship management (CRM) capabilities to meet the goals of providing more targeted promotions, improved understanding of regional demand, and improved product distribution. BlackBean wishes to include analysis of social media sources as part of its customer engagement reporting. This capability is required in order to leverage the customer relationships that BlackBean has already established into a sustainable competitive advantage. CRM capabilities are known to sustain and increase competitive advantage for small-to-mid-sized companies \parencite{pohludkaBestPracticeCRM2019}.

BlackBean also needs to address supply chain management (SCM) capabilities to aid improving product distribution and to ensure the quality and freshness of perishable items within the supply chain. An enterprise reporting platform (ERP) would not be unwarranted to help with improving business operations. Both of these capabilities are necessary to transform BlackBean into a data-driven company. The company already is trying to utilize data effectively, but the manual processes it engages in are inefficient and ineffective. Becoming an fully data-driven company will likely lead to sustainable long-term performance advnatgages for the company \parencite{ajimokoConsiderationsAdoptionCloud2018}.

Stil, BlackBean  has a number of constraints. The company is not interested in increasing its Information Technology in-house footprint, and has a small IT staff. The company seeks ease of use and ease of management, and has little appetite for significant disruption. Wishing to not take on more in-house IT capabilities strongly pushes towards Cloud Computing (CC) being a key component of the solution. This is itself a positive management goal, as small-to-mid-sized companies seeking SCM, CRM, and ERP capabilities are best served by utilizing CC \parencite{taliaCloudsScalableBig2013}. Modern SCM tools can be used to create significant efficiencies, reduce waste, ensure compliance, ensure freshness, and simplify SCM \parencite{aviles-sacotoGlanceIndustrySupply2019,daneshvarEffectiveFactorsImplementing2020,divaioBlockchainTechnologySupply2020,fossowambaDynamicsBlockchainAdoption2020,gaurBuildingTransparentSupply2020a}. Modern ERP tools will help with improved information quality, improved user satisfaction, improved organizational flexibility, increased efficiency, cost reductions, and improved decision making \parencite{hadidiSystematicApproachERP2017}.

It is worth noting that the reporting limitations and operational overhead of the current environment has become problematic. The sheer weight of manual effort makes further growth expensive due to needing to hire staff into headquarters to handle marketing and sales reporting, as well as creating a staffing demand for supply chain managers at effected regional operations centers. Utilizing CC services which scale very efficiently is a good first step towards addressing staffing impacts, however, the academic research on this impact is lacking \parencite{nieuwenhuisShiftCloudComputing2018}. It is, however, almost certain that a different set of skills would be needed from the IT staff than they likely currently possess \parencite{marquisImpactCloudComputing}.

\section{Recommendations}

First, BlackBean must recognize that it lacks the internal managerial knowledge and IT skills to address this problem effectively. Further, there are several key organizational criteria that are necessary to large-scale IT system implementation support. These include leadership knowledge of technology, leadership dedication to technological adoption, organizational capacity to acquire and infuse knowledge, and organizational agility \parencite{aliEmpiricalStudyExplore2019,bagKeyResourcesIndustry2021}. However, it should also be noted that technological tools empower and support innovation culture, and innovative companies will tend to remain successful in the marketplace longer than those who are not innovative \parencite{ahmadImpactWorkplaceInformation2020,nevoExploringRoleIT2020}.

To address this issue of managerial knowledge and commitment, the first step BlackBean should take is for their senior leadership to engage in ``boot camp'' type training for technology leadership. BlackBean must recognize that while they may sell tea and desserts as their products, as Accenture Group CTO Paul Daugherty said, ``Every company is a technology company'' \parencite{hristovEveryCompanyTechnology2020}. This will allow BlackBean to reframe the problem they are trying to address. BlackBean doesn't lack particular IT components or capabilities. BlackBean lacks a technology strategy. Instead of being concerned about their current employees being able to use or maintain a highly complex system, BlackBean needs to consider how it intends to operate in an increasingly dynamic, technology driven marketplace in the next 5, or even 10 years. Thus, a shift in managerial perspective is required.

Second, BlackBean should consider re-organizing its IT decision making and authority into its own department headed by an IT savvy CIO. Alternatively, IT could be re-organized under the COO at the director level. This move is necessary as the core function of IT will shift through BlackBean's strategic journey from supporting the accounting system to supporting all corporate, operational, and retail functions, and providing detailed reporting to support decision making at all levels of the enterprise. As such, the department must be properly situated in the organization with leadership at an appropriate level focused on technology strategy and technology execution. It must be understood that innovation in operations is positively correlated with performance in the retail sector, and that correlation is only likely to grow in the future \parencite{pintoInnovationStrategiesRetail2017}.

Once these two actions are taken, BlackBean must take steps to outline a ``cloud journey.'' BlackBean is correct that building out its own internal IT infrastructure is an inefficient solution to address their problem. Compared to Cloud-based Platform-as-a-Service (PaaS), Software-as-a-Service (SaaS), Infrastructure-as-a-Service (IaaS), and Desktop-as-a-Service solutions (DaaS), it is just too costly to staff for and house on-prem computing solutions. These service-based CC solutions have consistently shown themselves to reduce time to market, provide flexibility, provide better focus on core competencies, and provide superior access to specialized resources compared to non-CC solutions \parencite{wulfIaaSPaaSSaaS2021}. Based on this, there are two possible recommendations for BlackBean.

\subsection{Path A}

Based on BlackBeans' core market strengths and complex supply needs crossing multiple national boundaries, a cloud based, PaaS CRM solution such as Microsoft's Dynamics 365 or Salesforce Essentials, should be the first project tackled by the newly formed CIO team. BlackBean should plan on a 1.5 to 2-year long implementation plan. During this time, the IT team must be either up-skilled or restaffed to support a CC-first oriented company. BlackBean should contract with an experienced 3rd party consulting company to help with this transformational journey. Such an organization will bring both the CC technical knowledge, but also knowledge around change management, employee training, and other necessary components of a successful transition \parencite{farhanSystematicReviewDetermination2018}. BlackBean needs to tightly manage this relationship, as having a strategic technical implementation expert team as a partner can help with implementation success and long-term success.

For specific product recommendations, Microsoft Dynamics 365 should be considered the front-runner. The system is specifically designed to require minimal programming knowledge to support, to be easily configurable with minimal training; and, importantly, comes with SCM and ERP capabilities in extended modules.

This option will begin BlackBean's cloud journey and will give senior leadership insight into the capabilities and considerations around CC and PaaS systems. This path provides an immediate connection to the original irritation of poor customer reporting that BlackBean leadership identified. After the 6-month mark, BlackBean should have sufficient experience managing the integrator consulting company relationship to tackle ERP or SCM as the next step in their cloud journey. However, they shouldn't ignore the possibility that DaaS may be a better second step. DaaS implementations are fairly simple comparatively, and will sharply reduce the need for in-house call-center support for desktop problems. This could provide an immediate implementation win to the team.

\subsection{Path B}

This option considers BlackBean's lack of experience in managing integrator relationships, large-scale IT projects, and large-scale technology change. For this journey, BlackBean should start with a DaaS project to replace all end-user devices, including PoS systems, with advanced CC connected systems. This is a (relatively) low-cost, low-risk project. This will allow BlackBean's leadership to learn about managing large-scale IT transformation without having to commit to a multi-year project to see the full results. It will provide an opportunity to learn while having comparatively limited potential for making serious and consequential project management mistakes. Specific product options include DesktopReady by Anunta, Citrix' Citrix Workspace, Amazon Workspaces, Windows Virtual Desktop, or IBM's VMWare Horizon.

Specific recommendations should include Citrix, as a well-respected industry leader, and IBM's product, as IBM is also a well-respected system integrator. These products are, however, largely interchangeable, so at issue should be determined by total cost of ownership calculations driven by the RFQ responses.


While this transformation is underway, the company can be investigating SCM, ERP, and CRM service solutions in order to make well-informed decisions on product selection. Further, the DaaS solution will have the result of changing their staffing needs. This will provide them time to up-skill or re-staff their technology team to address their new strategic technology direction.

\section{Optimal Solution}

Given the above considerations, the recommendation is:


\begin{enumerate}
  \item Enroll all senior leadership in a high-quality technology boot-camp training course
  \item Begin evaluating international technology integrating companies for a strategic CC partner
  \item Make the decision to reorgnaize IT to its own entity, start the process, and begin searching for a CC knowlegeable CIO
  \item Hold an off-site strategic planning session focused on short and long term technology strategy
  \item Select an integration partner
  \item On the basis of the strategic planning session select a DaaS solution
  \item Establish and commence initial CC DaaS project
  \item Select CIO
  \item Continue to run DaaS project to completion while up-skilling / re-staffing IT
  \item Develop business requirements priorities for CRM
  \item Establish and commence Dynamcis 365 CRM module integration when CIO believe IT is ready to own managing the project
  \item Complete DaaS project (this should be roughly 6 months into CC journey)
  \item Continue CRM implementation
  \item Begin re-balancing marketing staff, and possibly retail staff, to take advantage of CRM efficiencies
  \item Begin evaluating e-commerce opportunities provided by enhanced CRM capabilities
  \item Create business requirements for SCM implementation
  \item Begin Microsoft Dynamics SCM implementation to integrate with existing CRM capabilities
  \item At the 1 year mark, hold a second strategic technology planning event for senior leadership, evaluating current position and future needs and direction

\end{enumerate}

This path addresses all identified structural weaknesses. It maximizes the likelihood of both short-and-long term success while minimizing risk. It does not include excessive lead-time to addressing the CRM requirements that have already been identified. It prioritizes creating an environment for success followed shortly by CRM needs. It allows for easy integration of CRM and SCM capabilities by using the same stack. It is entirely CC based and therefore limits or even decreases in-house IT demands. It recognizes the complexity of the work but off-loads that complexity to a strategic partner. It restructures the IT team to have a seat at the leadership decision making table to minimize the chances of BlackBean finding themselves lacking in critical technology infrastructure in the future. It recognizes that a CC journey is multi-year process and does not try to solve every issue quickly.

It is a demanding proposal in that it requires significant investment from senior leadership. However, senior leadership investment is a critical harbinger of success for technology innovation. By taking a CC solution seriously, BlackBean will be well-positioned for future expansion while ensuring the scalability, reliability, and capabilities of their technology stack. By facing the staffing re-skilling/re-staffing issue directly, BlackBean will ensure they have the knowledge in-house to properly manage their new capabilities.
