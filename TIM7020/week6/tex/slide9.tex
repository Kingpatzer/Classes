\begin{frame}
  \frametitle{Key Disadvantages}
  \begin{itemize}
    \item<1-> Resource constraints

          \note[item]<1-> {\scriptsize{The largest issue for microservice cloud architecture today is that Cloud computing in general faces a lack of skilled resources globally. The reality is that cloud computing requires a different set of skills than in-house IT staff usually posses \parencite{marquisImpactCloudComputing}. Companies moving to cloud-computing based microservices will need to address staffing gaps. Still, this is a gap that will need to be addressed regardless, as cloud computing is clearly the future of enterprise computing \parencite{gartnerGartnerForecastsWorldwide}.}}

    \item<2->  Complexity

          \note[item]<2-> {\scriptsize{Microservices themselves are individually simple, but by breaking systems out into multiple microservices, complexity overall may increase. This complex call-response relationship between microservices can generate its own challenges which must be managed \parencite{giovannitoffettiArchitectureSelfmanagingMicroservices2015,kangContainerMicroserviceDriven2016}.}}

    \item<3-> Costs

          \note[item]<3->{\scriptsize{While microservice development and cloud computing can to reduce costs overall, transitioning from traditional datacenter computing to cloud computing carries its own costs \parencite{sharmaPrioritizingCriticalFactors2020}. These costs can seem prohibitive if not viewed in the context of a comprehensive shift in a firm's long-term IT strategy and instead are taken as the costs of a single project. }}

  \end{itemize}
\end{frame}
