\begin{frame}
  \frametitle{Introducing Microservices \& Cloud Architecture}
  \framesubtitle{A New, Old Way to do New Things}
  \begin{itemize}
    \item<1->  Microservice Architectures are new, taking only \$2,073 million of the global IT market share of the \$3,700 Billion global IT market in 2018.

          \note[item]<1-> {\scriptsize{Total market penetration of microservice architectures is very small \parencite{rakeMicroservicesArchitectureMarket2020,gartnerGlobalInformationTechnology2021}. Firstly, it is a relatively new idea. First appearing in research publications in 2015 \parencite{lucaflorioDecentralizedSelfadaptationLargescale2015}, the architecture is gaining wide acceptance because of the many benefits. }}

    \item<2-> Benefits include scalability, performance, availability, security, and testability.

          \note[item]<2-> {\scriptsize{These benefits come from the nature of microservice architectures, which at essence is having each architectural component, such as identity management, or data-storage, perform only that single function. By keeping each element small and functionally focused, the system gains advantages over monolithic designs \parencite{liUnderstandingAddressingQuality2021}. }}

    \item<3-> ``Do one thing, and do it well.'' is part of the UNIX philosophy expressed in 1978. Microservices are similar and different from this idea.

          \note[item]<3->{\scriptsize{\textcite{mcilroyBSTJ57JulyAugust1978} expressed the UNIX philosophy. For UNIX this idea was based on composable small programs focused on programmatic tasks. Microservice architecture extends this idea from the small components being composable the level of programming tasks and suggests that systems can be composable of business-level services \parencite{redhatWhatAreMicroservices}. }}

    \item<4-> The lose coupling of services allows the benefits of the architecture to be most fully realized when combined with Cloud computing

          \note[item]<4->{\scriptsize{Cloud computing has many advantages similar to those of microservice architectures \parencite{ibmBenefitsCloudComputing2020}. These advantages have significant synergies when combined, and indeed, all of the major Cloud providers today (Microsoft, AWS, Google, Alibaba) provide easy tools for building microservice architectures.  }}


  \end{itemize}
\end{frame}
