\begin{frame}
  \frametitle{Lose Coupling}
  \framesubtitle{A Key Feature}
  \begin{itemize}
    \item<1-> Microservices communicate through APIs

          \note[item]<1-> {\scriptsize{This structure ensures that each component can be agnostic as to how it is built internally. This creates independence between components that simplifies and speeds development \parencite{WhatAPI}. It allows the development teams to make independent choices for implementation details such as language, OS, tools, libraries, and frameworks. This also can speed development time.}}

    \item<2-> APIs can be private, semi-private, or public, enhancing collaboration

          \note[item]<2-> {\scriptsize{With cloud computing and microservices, API can be restricted to only internal use, or access can be granted to specific business partners, or to the general public. This facilitates easy collaboration and enhances the value proposition for each component \parencite{WhatAPI,dasilvaMicroservicebasedMiddlewareCollaborative2020}.  }}

    \item<3-> Lose coupling ensures ease of development

          \note[item]<3-> {\scriptsize{APIs can be simulated, or substituted during development. This ensures independent development paths for each component, improving the ability of each team to deliver business functionality in a timely fashion.  \parencite{zimmermannMicroservicesTenets2017}.}}

  \end{itemize}
\end{frame}
