\begin{frame}
  \frametitle{Key Advantages}
  \framesubtitle{Security}
  \begin{itemize}
    \item<1-> Cloud tools provide best-in-class security practices

          \note[item]<1-> {\scriptsize{All of the major cloud providers provide best-in-class security practices such as well as best practice recommendations gleaned from dealing with security at scale, for example see \parencite{AWSSecurityBest}. This minimizes having to provide one's own security solutions to identity management, which is a notoriously complex problem.   }}

    \item<2-> Cloud providers ensure underlying systems are always patched to the latest security patches

          \note[item]<1-> {\scriptsize{A large part of exploiting systems in the wild is finding unpatched systems that are still vulnerable to published exploits. Cloud computing environments have dedicated large teams and automated tools to ensure the underlying systems are patched to the latest published patch levels at all times. This minimizes security management tasks for those using microservice Cloud architectures and maximizes security. Patching the systems running on cloud computers still remains the Cloud users' responsibilities, as AWS (2016) and others clearly explain.}}

    \item<1-> The small size of microservices make security vulnerabilities easier to detect and fix

          \note[item]<1-> {\scriptsize{Large-scale monolithic systems are notoriously difficult to debug \parencite{ahnScalableTemporalOrder2009}. Microservices, being focused on a single business functional component are smaller, and thus easier and faster to debug. This means that finding and fixing security flaws is simpler and less costly than in monolithic systems.}}


  \end{itemize}
\end{frame}
