\renewcommand{\mykey}{chengFacilitatingSpeedInternationalization2020}
\fakesection{\citeauthor{\mykey} \citeyear{\mykey} }
\fullcitebib{\mykey}

%% \end{adjustwidth}
%% Text here
The authors note recent research on the effects of business intelligence (BI) on international activity of businesses. Seeking to understand this link, they propose a theoretical model where BI impacts market capitalizing agility and operational adjustment agility. They then suggest that these two factors comprise organizational agility, and the speed of internationalization is driven by them mediated by cultural distance. They utilize firms in the Yangtze River Delta area of China to provide their research sample with governmental assistance. They perform reasonable tests for statistical validity, and proceed to test a collection of variables using multiple regression analysis techniques. They find all their hypotheses hold. Overall, the study is unconvincing. First, the reliance on government officials to drive question answering opens up serious questions about data validity. Second, connections between the factors seems contrived. While indubitably important factors, there's no proposed mechanism for how BI can facilitate operating in multiple languages for example. Third, the study is done with firms located in an area where the Chinese government has established business parks that provide grants for BI projects and international expansion. Finding no correlation between the variables here would be remarkable. Lastly, the statistical methods seem suspect and read as if the analysis was done to support the conclusion. It doesn't read as the proposed analysis preceding the modeling. However, this last point may be a factor of language barriers and not an actual limitation.

\newpage
%% NOTHING FOLLOWS
