\renewcommand{\mykey}{souibguiDataQualityETL2019}
\fakesection{\citeauthor{\mykey} \citeyear{\mykey} }
\fullcitebib{\mykey}

%% \end{adjustwidth}
%% Text here
The authors present a comprehensive review of data quality (DQ) concerns and approaches to ensuring DQ in a Big Data environment. The authors note that DQ can be addressed through process oriented or data oriented approaches. They define and substantiate 15 dimensions of data quality and show how each aligns  to different steps in a BD extract, transform, load (ETL) process. The authors then experiment on 4 different ETL tools to show how they handle common DQ issues. The study effectively highlights both the importance of DQ for BD applications and business value, but also effectively demonstrates how modern current tools only partially help with DQ in common ETL scenarios. The study notes that DQ can be addressed based on semantic, syntactic, and contextual levels. Based on an examination of popular tools and the existing literature, the authors point out that contextual DQ tooling is lacking, as is the research into how to address DQ with contextual information.  The study serves as an excellent reminder of the underlying complexity of BD applications, and the degree to which tooling solutions to common problems need to go in order to be truly valuable. As presentation of demonstrated results, the paper has no particular weaknesses, and is well sourced with relevant citations around BD, DQ and ETL. It is of course limited in that it is properly a literature review and a test of four specific tools and not a comprehensive review of all available tools. However, there is no reason to believe the presented results are not adequate representatives of the state of currently available tool sets for DQ in BD ETL settings.

\newpage
%% NOTHING FOLLOWS
