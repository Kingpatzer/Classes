\renewcommand{\mykey}{lopez-robles30YearsIntelligence2019}
\fakesection{\citeauthor{\mykey} \citeyear{\mykey}}
%%\begin{adjustwidth}{1in}{}
%%  \begin{hangpara}{.25in}{1}
    \fullcitebib{\mykey}
%%   \end{hangpara}
%%\end{adjustwidth}

%% \end{adjustwidth}
%% Text here
%%
    This study looks across multiple research areas where the term ``Intelligence'' is used from 1987 to 2017. The authors note that across research domains, the term has a plurality of uses that do not always align. This study, which the authors assert is the first of its kind, utilizes common meta-analytical methods for categorizing themes and trends in basic research. The authors plot keyword bundles along 2-axis, centrality and density. Centrality is the degree of interaction with other theme networks, and density is the internal strength of the network. The research also classifies the importance of individual publications and authors by tracking citations, and utilizes the well-established H-Classics method. It is a well-structured meta-study. This paper is highly valuable for understanding the history of related research across domains.
   It explicitly points to the ``terminological plurality'' that the term ``intelligence'' holds within various research groups. It serves to contextualize ``business intelligence,'' among other terms in a broader context and does so well. The paper is well written, has numerous informative graphs, and is a good general introduction to the progression of these research topics. It also offers likely areas of future research growth related to intelligence. The paper is also, frankly, a master class in the use of tables and graphs to illustrate complex talking points. The major limitation of the paper is that it goes back only 30 years, while the first use of ``Business Intelligence'' as a research topic comes from the 1950s.

\newpage
