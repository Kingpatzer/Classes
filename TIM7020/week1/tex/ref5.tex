\renewcommand{\mykey}{jiModelingImageVideo2019}
\fakesection{\citeauthor{\mykey} \citeyear{\mykey} }
\fullcitebib{\mykey}

%% \end{adjustwidth}
%% Text here

This paper provides a case study of researchers building a quality management data analytics system for an aerospace manufacturer in China utilizing data warehousing (DW) and business intelligence (BI). The paper provides a good overview of both BI and DW but is light on citations. It seems aimed at practitioners more than researchers. The implementation chosen, a DW based on an OLAP cube with a single fact tables for each different product is not unique or new. Layering BI capabilities on top of an OLAP cube using off-the-shelf Microsoft components such as MS PowerPivot 2013 is a fairly standard implementation. The title suggests that the BI system is purposed to deal with unstructured data in a complex environment. However, the authors go into depth on how the OLAP cube is highly structured data as one would expect with most OLAP implementations. This is an interesting case study only in that it provides real insight into some common pitfalls with traditional OLAP/BI system design choices such as making informed decisions about fact tables and dimensions during system design. However, it fails to deliver on the promise of its title. A case study of a current-day implementation of a 7-year old BI tool set should have little impact on current researchers. However, it still has utility as a reviewed academic source to point practitioners to when discussing how to architect and implement such systems successfully. This is particularly true given the scale of the system in question.

\newpage
%% NOTHING FOLLOWS
