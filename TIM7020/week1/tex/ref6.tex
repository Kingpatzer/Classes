\renewcommand{\mykey}{lozadaBigDataAnalytics2019}
\fakesection{\citeauthor{\mykey} \citeyear{\mykey} }
\fullcitebib{\mykey}

%% \end{adjustwidth}
%% Text here

This study presents a look at co-innovation and big data analytics (BDA). The authors surveyed a large number of firms in Columbia to test a model that relates a firms tangible, human, and intangible resources to its BDA capabilities, and then hypothesizes that BDA capabilities will help drive co-innovation. Co-innovation is defined in the paper as participation of internal and external stakeholders in a company's development of new services, products, and even business models. The survey results are analyzed for validity. The hypotheses presented held and the model was able to explain 63\% of the observed variability. These results are interesting given that Columbia is a developing economy and is generally not considered a technology leader. If the results can be replicated in other locales, the authors may have helped in providing insight into how to rapidly develop an innovation culture in a business network in developing nations. The research focuses on an area of emerging interest, as little research on co-innovation exists, and its relationship to BDA capabilities are not well understood. It will be interesting to see if anyone takes up the challenge to see if these results replicate in different areas with differing levels of economic development and technological advancement. It seems probable that the overall idea can be generalized, but that the weights of the various model elements would change depending on contexts.

\newpage
%% NOTHING FOLLOWS
