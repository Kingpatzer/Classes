\section{Literature Review}

While BD\&A, Blockchain, and AI are among the fastest areas of computer growth, facilitated and driven by cloud computing /parencite{gartnerTrendsImpactingCloud2020}, with large-scale integrators investing billions to support cloud-based services for clients \parencite{accentureAccentureCloudFirst2020,taliaCloudsScalableBig2013}, adoption lags for SMEs. These SMEs face a serious strategic challenge in determining when and how to develop BD\&A capabilities. This challenge is primarily around balancing the extensive up-front costs and risks with the perceived long-term benefits \parencite{ajimokoConsiderationsAdoptionCloud2018}.

It is known that data-driven companies are more profitable then their competitors \parencite{alsghaierImportanceBigData2017}. But researchers have noted that there is little understanding as to how BD\&A and related investments help some data-driven companies become successful but not others \parencite{morenoBusinessIntelligenceAnalytics2020}. The crux of the issue is that while research has also demonstrated that becoming data-driven is necessary for long-term competitive advantage \parencite{alsghaierImportanceBigData2017} the mechanism for translating BD\&A capabilities into corporate success has not been an adequate focus of research \parencite{trieuGettingValueBusiness2017}. And in some business domains, the business leaders even disagree with CS experts that there is much more value to be gained by increased technology investment. \textcite{rossmannFutureSocialImpact2018}, used a delphi research methodology to gather a large, multi-domain, heterogeneous collection of 73 experts from both academia and business. They investigated the realm of SCM and how BD\&A and other associated technologies would impact the supply chain in the future. The experts had some interesting points of agreement, but the more interesting results is that business SCM experts simply don't see the new technologies as being a reliable producer of value for the businesses. The basis of this disagreement in the SCM space rests on the fact that supply chains are already extremely lean and some amount of laxity in the supply chain is always necessary to be able to provide resilience in the face of any number of unforseeable events.


Value creation has been recognized as coming from firm characteristics that above and beyond IT investment \parencite{bozicBusinessIntelligenceAnalytics2019}. Thus, SMEs are faced with a ``no-win'' situation wherein they are being asked to expend capital, embrace technological uncertainty, and take on associated risk without clearly proven demarcations of success. One are where non-IT investment is known to matter to the success of BD\&A and AI investments is around the various dimensions of data quality. \textcite{souibguiDataQualityETL2019} outlined 15 dimensions of data quality and show how each impacts BD\&A and AI viability utilizing 4 different ``extract, transform, load'' tools. This research importantly demonstrates that investment per-se isn't always the limiting factor to success as much as matching the investment to the characteristics of the problem that must be solved.

Further, research such as that done by \textcite{bozicBusinessIntelligenceAnalytics2019} show that firm characteristics such as absorptive capacity, the ability to innovate both radically and incrementally at the same time, is a key characteristic linked to successful BD\&A adoption. But studies such as this are both relatively new and relatively rare in the research literature. Further, this study is limited to firms in Slovenia, and there is ample reason to question if there might not be culture confounding variables that would make the results less portable across cultural boundaries.
