\section{Goal}

Ideally, the problem statement could be answered with a definitive set of questions which if answered in the affirmative would provide a practitioner a guaranteed probability of success. However, this is highly unlikely as a plausible outcome for this type of research. The predictive power of social science research, such as identifying organizational characteristics, has been long known to be limited \parencite{KimballYoungMethod}. This has not changed greatly over recent times \parencite{taageperaMakingSocialSciences2008}. Moreover, the typical method of acquiring information about organizational characteristics is through methods such as structured interviews and surveys. Such interactions have a notoriously low response rate from senior leadership, meaning the resulting data is often fairly small with marginal effect sizes. Such self-selecting surveys also suffer from other potential statistical defects \parencite{zikmund1997business}.

In order to overcome these issues, this proposal has the specific goal of being able to ``survey'' thousands of SMEs leadership characteristics without their being overtly aware that they are being studied, and without them having the ability to self-select out of the study. To accomplish this ethically, it is of course necessary to rely entirely on publicly available data. Thus the central goal of this study is to utilize Big Data \& Analytics methods of textual analysis to identify SMEs that are engaging in a SCM BD\&A, Blockchain, or AI transformation by automatically analyzing SEC filing and press releases for key phrases. Then, using similar textual analytical tools, the system will use publicly facing data from social media sites such as Twitter, Facebook, Linked-In, Glassdoor and others to identify employees and analyze two key dimensions: their perception of key leadership characteristics, and their level of engagement. This study will track the success of the SME implementation effort again through press releases, SEC filings, and other public statements. Finally, financial performance and market share of the companies will be evaluated for the year period prior to the start of the SCM journey and for two years after the initial launch.

The companies collected will be broken into three sets randomly, a training set, a validation set, and a research set. The training set will be used to train the model. The validation set will be used to determine the expected performance of the model, and the research set will be used to make actual predictions and test them against observed outcomes.

Success for this study will be measured in three ways. First, the development of a predictive data mining model that can determine the likelihood of success with confidence interval superior to random guessing. Second, a set of categorized organizational characteristics which allow a description of SMEs who are ideally suited for SCM BD\&A success. Third, the ability to built a descriptive matrix of organizational characteristics, employee perceptions of leadership characteristics, and employee engagement that mediate SMC BD\&A strategy success.
