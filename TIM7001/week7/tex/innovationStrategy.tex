\section{Deployment Strategy}

We have seen that cloud capacity, internet capacity and stability, cultural norms, psychological acceptance, and regulatory burden are all potential barriers to a global deployment of innovative technology such as blockchain for SCM. Of course, none of these are in and of themselves an absolute impediment. However, any company that wishes to rise to the challenge must have a strategy for facing each of these challenges in turn.

Let's start with some observations about the ideal way to deploy technology of the scale of an SCM blockchain system. Such systems inherently benefit from a scalable architecture \parencite{gokalpAcceptanceBlockchainBased2019}. SCM is a resource heavy system in the first place, and blockchain capabilities only increase those resource demands. These resources demands are quite expensive but the benefits of SCM and blockchain integration exceed the deployment costs when managed well \parencite{balSupplyChainFinance2021}. The most cost effective method for building large computer systems and deploying them globally is to utilize cloud service systems like Amazon's AWS, MIcrosoft's Azure, or others \parencite{carcaryAdoptionCloudComputing2014}. This works quite well in locations that have adequate cloud computing centers either located near the company, or connected to the company with low-latency, high-bandwidth internet connections. However, as we have seen, in locations such as Africa there is a distinct lack of established cloud infrastructure.

This presents two options to the company wishing to address the problem this creates. The first is to mitigate the impact by selecting suppliers who are located near appropriate infrastructure. Most resources are available from multiple suppliers in multiple locations. Very few resources available from Africa are not available from any other continent. So one very real strategy is to simply find a new supplier that is located in a location more amicable to cloud computing. A second strategy is to partner with the other companies in the supply chain to build physical non-cloud infrastructure to serve the supply chain to augment any cloud connectivity that is not locally available.

When it comes to addressing the concerns of network bandwidth, these same two options exist. For network connectivity, satellite network connections can supplant the poor to non-existent network infrastructure. This technology is not inexpensive, but if changing suppliers is not possible it may be the only real option to ensure the needed bandwidth is available in the proper locations. As the advantages of a blockchain enabled SCM system permeate the entire supply chain, if changing suppliers is not an option and the strategy of building the necessary infrastructure must be followed, then forming a limited partnership with other supply-chain participants to fund, staff and operate any locally constructed networking or data center capacity is appropriate.

Addressing cultural norms is not as simply as devising a solution to technical infrastructure limitations. As seen from the discussion about the Greek shipping industry, cultural norms can drive non-adoption. In the case of the Greek shipping companies, the benefits of adoption are well known in the shipping industry \parencite{groenfeldtIBMMaerskApply2017}, but the Greek shipping companies still resist. In order to address cultural barriers to adoption, a company must engage in a significant communication and information sharing strategy to overcome hesitation to adoption. Research has shown that culturally bound perceptions will change in the face of significant, consistent communication efforts entered into in a good faith effort to inform and empower \parencite{gefenTrustTAMOnline2003, parkUserAcceptanceDigital2009,dubeySwiftTrustCommitment2019}.

These same techniques are necessary to address psychological barriers to adoption. Indeed, cultural barriers are, essentially, simply shared psychological barriers. So, in addressing one, it is possible to address the other. However, there is a slight difference in that psychological barriers tend to be tied more to individual perceptions around complexity and ease of use rather than perceptions around relational constructs. The Greek shippers avoid blockchain not only because of perceptions about difficulty in using it but because of a shared commitment to the cultural norms of doing business in person. To address psychological barriers to adoption, companies need to develop targeted training for participants in the system to demonstrate both ease of use of the product developed and the utility of the system for them in their particular tasks \parencite{louIntegratingInnovationDiffusion2017}.

Regulatory differences between countries, regions, and local areas can present an even greater challenge than any of the above options. For companies that are already engaged in global business operations, this means starting early to work with the various legal departments in each country and region to ensure that requirements and potential areas of conflict can be identified early and addressed in the system design. For example, if since personal identifying information can not be removed from the blockchain as the blockchain is immutable, one plausible solution to avoid privacy issues in the future is to design the blockchain to carry not personal identifying information, but a key to the database that stores personal identifying information (PII) separate from the blockchain ledger. This will allow the PII to be treated differently depending on the regulations which apply, will allow for PII anonymization, and other steps to address potential regulatory concerns. But it is necessary that the system architects are aware of the legal and regulatory demands early so that solutions can be devised.
