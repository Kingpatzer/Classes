\section{Deploying Innovation}

Deploying innovative technologies can face many challenges. Some of these challenges are based on psychology and culture, while others are based on technological issues. Still other issues may be regulatory or legal in nature. In considering rolling out a new technology in a global enterprise, it is important to consider how these challenges may impact both local and global costs to deployment, and how they may or may not impact the probability of success. Not all challenges are insurmountable, but some are extremely difficult to overcome, and as such must be recognized early to avoid overly investing in a poor technology strategy.

For the purposes of this paper, we will consider the issues faced by a large, multi-national company based in the United STates wanting to deploy blockchain technology in support of a supply chain operations both locally and globally. For this deployment challenge, not all issues will impact global and local deployments equally. Indeed, some will barely constitute a problem in one context and will be nearly insurmountable in others. However, each issue is one that needs to be considered in the overall deployment effort.

\subsection{Technological Factors}

From the perspective of technological factors, there are many that can be considered. But in this instance we will focus on just a few. The first issue that must be considered is cloud infrastructure availability. As some of the uses of blockchain is to help ensure sourcing and labor practices to both appease consumer demand and to meet corporate ethical standards \parencite{kouhizadehBlockchainTechnologySustainable2021,saberiBlockchainTechnologyIts2019}, it is essential that blockchain technology is supported by the entire supply chain. This end-to-end support is what provides for a transparent supply chain for both the corporate entities involved and the consumer \parencite{gaurBuildingTransparentSupply2020a}.

For our global corporation, Africa, as well as other less-developed nations in South America and Asia, presents a significant source of resources for our supply chain \parencite{woodAfricaExportStructure2001}. However, Africa and other less developed regions of the world, significantly lags behind industrialized and post-industrial nations in cloud infrastructure and the cost of that infrastructure is significantly more. \textcite{mckinseySolvingAfricaInfrastructure} found, for example, that generator-based power costs three to six times that of grid-based power in other locations within Africa. For our use-case, cloud infrastructure is nearly absent. While Africa is the second largest continent, representing more than 20\% of all the earth's landmass, it contains only one Amazon Web Service region, comprised of only 2 availability zones (AZ), both located in the far south \parencite{awsGlobalInfrastructureRegions}. By way of comparison, Europe has six regions and at least 16 AZs.

Because one of the easiest ways to solve global deployment challenges is with cloud technology, a point that will be discussed in depth later, this presents an extremely difficult situation for global companies. It is not really possible to locate the needed computing infrastructure within easy reach of the source of materials. This means that anyone wishing to build end-to-end supply-chain blockchain systems in Africa need to invest heavily in private computing facilities to support the computational needs of the blockchain systems.

However, computing power is not the only infrastructure lacking. Blockchain also consumes a large amount of network bandwidth \parencite{litkeBlockchainsSupplyChain2019}. This requires modern networking infrastructure. While there is fairly respectable internet penetration into Sub-Saharan Africa. The quality of service is not comparable to that seen by more modern regions. It suffers from two major problems. The first, is that ownership interests of authoritarian regimes means that internet connectivity can be, and is, shut down for political motivations. Additionally, the quality of service is lower than typically requierd to support the type of high-bandwidth usage that modern blockchains supply-chain systems require \parencite{freyburgBlockingBottleneckInternet2018}. Further, the political nature of communication infrastructure in Africa means that instead of having regional internet providers, there are ``scattered islands'' of connectivity \parencite{fanouReshapingAfricanInternet2017}. This results in a situation where even if adequate connectivity exists at two locations, unless those locations are controlled by the same ISP, under the same regime, there is a good chance that communication between sites will be spotty at best.

As blockchain is fundamentally predicated upon near-real time information sharing between trade partners, the inability of companies to be secure in political environment that protects open communication means that companies face serious difficulties in rolling out these types of new innovations into the global market space. What will work flawlessly in a place like Europe or the United States can be quite hampered by the ``digital divide'' that separates modern digital nations from less developed ones.

\subsection{Cultural and Psychological Issues}

Many barriers to technological implementation are not merely technological, or even have much to do with technology at all. Often the barrier to entry is almost purely psychological and cultural. A prime example of this phenomenon with respect to culture is the response of Greek shipping companies to blockchain supply-chain technology. The utilization of blockchain in the shipping industry to increase efficiencies, drive profits, and provide a significant competitive advantage is well documented. Maersk is leading this effort across the globe and their successes have been stellar \parencite{groenfeldtIBMMaerskApply2017}. Yet, despite the proven success of the world's largest shipping company in this arena, and a rising demand among shipping customers, Greek shippers collectively own the 22\% of the global share of carriers and 33\% of the tankers. This collectively is more than 50\% of the European Union fleet capacity. The industry is financially stable and successful, with the financial resources to embrace high-tech solutions \parencite{papathanasiouNonApplicationBlockchain2020}. However, cultural barriers persist that prevent this adoptions. Greek carriers rely on old-fashioned ``face-to-face,'' or at least ``phone-to-phone'' conversations and personal relationships to create shipping contracts. Their culture strongly and deeply embraces the roles of middlemen and personal contacts that blockchain is specifically tailored to eliminate as ``inefficiencies'' (ibid.). But where many see inefficiencies, Greek shippers see a strength. They believe their deep personal relationships create bonds that ensure continued business, minimize conflict, and allow for easy resolution of disagreements. Further, where companies like Maersk see sharing of information and transparency as an opportunity to eliminate economic waste, Greek shippers see information transparency as a threat to their way of business.

These deep cultural barriers are not easily overcome. Moreover, they can be an absolute barrier to market entry for new technology if business partners in the supply chain simply refuse to adopt the new and innovative measures because of cultural reasons.

From a psychological perspective, the literature is rife with evaluations of models for innovation acceptance such as the technology adoption model (TAM) which highlights that perceived utility and perceived ease of use are key factors in determining the rate of technological adoption. When a new, innovative technology is presented, the degree of acceptance and success it sees is strongly correlated to the psychology perception of the managers and workers who are asked to adopt the technology \parencite{gefenTrustTAMOnline2003}. Specific research into blockchain supply-chain systems in under-developed locales (namely India) showed that these psychological factors account for nearly 70\% of the behavioral intentions of participants \parencite{kambleUnderstandingBlockchainTechnology2019}. This is a highly significant figure and needs to be taken into account. Of the factors addressed so far, this is the first that applies universally to local and global contexts. There is no particularly context that strongly favors and disfavors consideration of psychological impacts to managers and workers when rolling out new technology.

\subsection{Regulatory Misfit}

Another area of difficulty for globalization of innovative technology is the impact of disparate regulatory structures in various locations. One easy to understand example of this is the European Union's General Data PRotection Regulation (GDPR). The GDPR requires several constraints that are not enforced in other areas, such as the United States. Specifically, the GDPR requiers consent of the subjects for data processing, anonymizing collected data to protect privacy, safely handling the transfer of data across borders, and having data officers to oversee data compliance. Of these, the consent for data is the most problematic for blockchain applications. Under Articles 17 \& 18 of the GDPR, individuals have, under certain circumstances the right to have their data removed \parencite{Art17GDPR}. However, one technological function of blockchain is that \textem{all data is immutable}.

This doesn't itself present an insurmountable hurdle to corporate-to-corporate blockchain supply-chain management systems. However, it does serve as a barrier to allowing small suppliers who are functioning as sole-proprietorship into blockchain SCM systems. This is because personal data would not be removable as a result of the technology itself.

Many regulatory incompatabilities exist between nations. Normally, this doesn't present a significant problem, but with electronic technology, where data collected in one country may be processed and stored in another country out of necessity, it can become a significant burden to ensure compliance with the rules. The complexity of the rules does result in companies with significant legal staff facing massive fines for falling afoul of the laws. Recently Google was fined almost \$60M by France for GDPR violations.
