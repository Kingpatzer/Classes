\section{Overview of Concept Map}

The key structural components of the concept map, the central point, is that Supply Chain complexity is a business problem that can be managed by blockchain technology. Blockchain technology, however, faces challenges for technology adoption because it has low perceived ease of use. Blockchain adoption is important for supply chain management not merely because of complexity, but it also strongly empowers a sustainable supply chain, which has growing business importance.

Diving deeper, the key point about complexity is that it drives costs and increases risks. This growing complexity comes from many points, including electronic communications \parencite{bojnecImpactInternetManufacturing2009}, globalization \parencite{fujitaGlobalizationEvolutonSupply2006}, and corporate actions \parencite{linawatiSupplyChainFlexibility2017}, just to name a few. But regardless of the cause, complexity increases costs and risks \parencite{thomasCostsBenefitsAdoption2016, wangRiskTransmissionComplex2021}.

This complexity must be managed. Blockchain provides a tool to manage this challenge. The very nature of the tool provides a number of capabilities that are necessary to the reduction of costs and risks. It improves trust in multiple ways \parencite{sahaiEnablingPrivacyTraceability2020, liuPrivacyProtectionFog2020, chanSimpleScalableBlockchain2021}, which lowers risks. Further risk reductions arises from time reduction \parencite{ivanovImpactDigitalTechnology2019}, transparency \parencite{pavlovHybridFuzzyProbabilisticApproach2018}, and enhanced record keeping \parencite{ibmHowBlockchainCan2016}.  It also directly addresses costs by improving billing accuracy \parencite{groenfeldtIBMMaerskApply2017} lowering transaction costs \parencite{schmidtBlockchainSupplyChain2019}, and providing disintermediation \parencite{crosbyBlockChainTechnologyBitcoin2016}.

Blockchain also helps with sustainability. Many of the same traits that address complexity also allow to manage a more sustainable supply chain. Traceability helps customers trust that the supply chain is sustainable \parencite{saberiBlockchainTechnologyIts2019, wangUnderstandingBlockchainTechnology2019}. Security ensures that the data is accurate (ibid.), and Immutable data in a reliable system is needed to drive sustainability (ibid).

However, all of these highly valuable benefits are not without issues that must be addressed. Importantly for technology adoption, as outlined by the technology adoption model (TAM), are perceived utility and perceived ease of use \parencite{parkUserAcceptanceDigital2009, gefenTrustTAMOnline2003}. While the perceived utiltiy for blockchain is quite high, particularly as it relates to sustainability \parencite{kambleUnderstandingBlockchainTechnology2019a, louIntegratingInnovationDiffusion2017}, the perceived ease of use of blockchain technology is low \parencite{kambleUnderstandingBlockchainTechnology2019a, yangMaritimeShippingDigitalization2019}. So, TAM suggests that adoption will be lower due to these perceptions.


\section{Key Insights}

The key insight that this concept map highlights is that two critical business problems (complexity and sustainability) are addressable through a single technological tool, but that tool is facing an adoption challenge because of perceptions of ease of use, even though the perceived utility of the tool is quite high, particularly as it relates to sustainability. Moreover, the same factors of blockchain that reduce cost and complexity of supply chain management are precisely the factors that are needed for sustainable supply chains.

Given the critical business importance of both items, this should drive leadership in companies that want to be sustainable to address perceptions of ease of use. Additionally, this should motivate producers of blockchain SCM tools to focus on user experience factors and to simplify the adoption and modification process of deploying SCM tools.

\section{Connection Between Theory and Practice}

The relationship between research and practice in business is frought with examples of areas where research and practice seem to have little to do with each other \parencite{brettleAreWeBridging2012, bansalBridgingResearchPracticeGap2012}. However, this is not an instance where this common phenomenon applies.

There are two places with theory and practice intersect. The first is that the theoretical advantages of blockchain are being shown in the ``real world'' to provide the benefits promised. These benefits arise both in the realm of address complexity and supporting sustainability. With respect to complexity, very large companies have been adopting blockchain SCM systems to specifically manage complexity and thus reduce risk and cost \parencite{groenfeldtIBMMaerskApply2017}. With respect to sustainability, the link with blockchain technology in real world sustainable supply chains is well documented \parencite{saberiBlockchainTechnologyIts2019, teixeiraHowMakeStrategic2019}.

The second clear intersection with theory and practice is that TAM suggests that perceptions of utility and ease of use mediate technology adoption. This too is being seen in practice as adoption of blockchain the SCM space lags expectations. The low rate of adoption of blockchain into the supply chain is caused by many different impediments, ranging from a lack of standards to costs of implementation \parencite{kristofferfranciscoSupplyChainHas2018}. However, a major driver of the low adoption rate is mediated by the perceived difficulties of building and using the system \parencite{papathanasiouNonApplicationBlockchain2020, kambleUnderstandingBlockchainTechnology2019a, partalaUnderstandingMostInfluential2015}.

\section{Lessons for Visionary Leadership}

The link between sustainability practices, supply chain, and blockchain technology are links that were created due to visionary leaders thinking about how to solve very particular problems. Blockchain was not even developed with the idea of the supply chain as an application \parencite{hughesBitcoinWhatBlockchain2019}. Yet, the numerous features provided by the technology was leveraged to extend beyond mediating financial transactions to mediating entire business relationships in an end-to-end sustainable supply chain.
