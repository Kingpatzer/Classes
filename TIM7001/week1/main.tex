% Created 2020-12-20 Sun 15:41
% Intended LaTeX compiler: pdflatex
\documentclass[man]{apa7}
\usepackage[utf8]{inputenc}
\usepackage[T1]{fontenc}
\usepackage{graphicx}
\usepackage{grffile}
\usepackage{longtable}
\usepackage{wrapfig}
\usepackage{rotating}
\usepackage[normalem]{ulem}
\usepackage{amsmath}
\usepackage{textcomp}
\usepackage{amssymb}
\usepackage{capt-of}
\usepackage{hyperref}
\shorttitle{BLOCKCHAIN AND ETHICAL SCM}
\usepackage{hyperref}
\usepackage{fontawesome}
\usepackage{csquotes}
\usepackage{hhline}
\usepackage{colortbl}
\usepackage{arydshln}
\usepackage{caption}
\usepackage[utf8]{inputenc}
\usepackage[gen]{eurosym}
\usepackage[style=apa,sortcites=true,sorting=nyt,backend=biber]{biblatex}
\DeclareLanguageMapping{american}{american-apa}
\addbibresource{/home/david/Documents/School/References/bibliography.bib}
\affiliation{North Central University}
\leftheader{Wagle}
\authornote{
\addORCIDlink{David A. Wagle}{0000-0001-8130-4900}
\hspace*{1.69in} \href{https://www.linkedin/com/in/davidwagle}{\faLinkedinSquare \hspace*{2pt} https://linked.com/in/davidwagle}

Correspondence concerning this article should be addressed to David A. Wagle, School of Business, North Central University, C/O PO Box 1997, Burnsville, MN, 55337.
E-mail: \href{mailto://david.wagle@gmail.com}{David.Wagle@gmail.com}}
\abstract{This paper examines a number of scholarly and general publications on the topic of blockchain as it relates to sustainable/ethical supply chain management issues.}
\keywords{Blockchain, Supply Chain Management, Business Ethics, SCM }
\author{David A. Wagle}
\date{\today}
\title{An Examination of Current State of Blockchain Technology in Ethical Supply Chain Management}
\hypersetup{
 pdfauthor={David A. Wagle},
 pdftitle={An Examination of Current State of Blockchain Technology in Ethical Supply Chain Management},
 pdfkeywords={},
 pdfsubject={},
 pdfcreator={Emacs 26.3 (Org mode 9.5)}, 
 pdflang={English}}
\begin{document}

\maketitle


\section{Introduction}
\label{sec:org938be51}

On December 14, 2020, \textcite{reutersAppleProbesViolence2020} reported that
thousands of employees of the Taiwanese company Wistron Corp. rioted in India
causing millions of dollars in damage. The workers rioted, ostensibly due to
forced reduction in salaries to less than \$7 US per month for some employees
\parencite{amadeoIPhoneFactoryWorkers2020}, as well as unsafe working conditions. While
India is an under-developed nation, this treatment of workers has caused
Wistron's major downstream customer, Apple to respond by sending auditors and
inspectors to India to examine the events
\parencite{reutersAppleSupplierPuts2020}. While this is not the first time that a
large corporation has had to deal with the fallout from ethical lapses in their
supply chain, these global-news grabbing events in some ways cover up the
greater issue that most corporations ignore corruption and ethical issues in
their supply chain \parencite{webbTwoThirdsCorporationsIgnore2017}.

But this intentional ignorance is problematic in a world where customers want to
purchase from companies that ethically source their products
\parencite{howellsSAPBrandVoiceEthics2020} and where the freedom to publish by anyone with
a smartphone means leaders can no longer control information release
\parencite{davenportGoogleHowTechnology2014}.

Enter blockchain. Blockchain gained its fame as the underlying technology behind
bitcoin. The blockchain system was originally conceived as a means of conducting
financial transactions securely without needing intermediaries and which
eliminates the ``double-spending'' problem while being able to operate without
requiring trust between parties. In one of the great moments of acute irony
in technology history, this trust-less protocol's author published anonymously
using the pseudonym Satoshi Nakamoto
\citeyear{nakamotoBitcoinPeertoPeerElectronic2008}. 

It did not take long for technologists to start noticing the potential utility
of blockchain to solve issues of trust in supply chain management. The central
characteristics of blockchain make it a potentially ideal system for helping
ensure ethical supply chain management. However, the rate of adoption, and
potential hurdles mean that adoption of the technology will require visionary
leadership across multiple organizations, as well as solving multiple
engineering hurdles in the process. 

\section{Blockchain Briefly Explained}
\label{sec:orgfafa325}

Blockchain's most prominent feature is that of the distributed ledger. Each node
in a peer-to-peer network contains some, to all, of the available nodes which
make up the chain. Each node's authentication results via cryptographic
public-private key hashes which demonstrate proof that a particular member of
the peer-to-peer network is responsible for the content of that specific node.
Trust in the content of anyone node is a product of the peer-to-peer structure,
where the lengthiest chain with peer consensus becomes the authoritative version
of the chain. Some blockchain protocols allow for the execution of associated
actions verified by the blockchain as part of a node. This allows
for ``smart contracts'' whereby adding a node, requires agreeing to contractual
terms and/or proof of contract execution attach to the ledger itself. Due to the
unalterable nature of the blockchain, enforced by the sequence of cryptographic
hashes, specific actions and data are thereby secured into the blockchain as a
permanent record \cite{mearianWhatBlockchainComplete2019}.

These properties make blockchain extremely attractive to solve various business
problems, provided a framework is in place to support adoption
\parencite{iansitiTruthBlockchain2017}.

\section{Blockchain and Supply Chain Management}
\label{sec:org527033c}

One area blockchain shows great promise is in helping businesses ensure
compliance with ethical sourcing standards. For example, if Apple's delivery
contract with Wistron included a mandate to include verified proof of proper
salary payments to employees, then Apple would not be facing the PR issues it
now is dealing with as a result of Winstron's failures in India. However, much
of the research on using blockchain technology to ensure ethical supply chain
management is more theoretical than practical
\parencite{venkateshSystemArchitectureBlockchain2020}. Indeed, an empirical
examination of the ``Fortune 500'' firms web-presence for information about
blockchain found that only 21, or 4\%, of the firms had a referenced use for
blockchain on their publically facing materials. Only a few of these 21
companies were using blockchain to improve supply chain information
\parencite{caseBlockchainEmpiricalReview2020}. This study used an examination of
the available publically facing webpresence of a company looking for the term
``blockchain,'' so it is limited in its applicability. However, given the theoretical
power of blockchain to function as a  transformative technology, it is a
reasonable assumption to expect that companies with major blockchain initiatives
would advertise that fact. 


\section{State of Research}
\label{sec:orged5ec5c}

One of the most important areas of research on blockchain technology with
regards to ethical supply chain management is in the arena of transparency.
While there is very little empirical studies of actual real-world
implementations of blockchain solutions, researchers have examined areas where
blockchain could solve numerous problems. For example,
\textcite{thakurLandRecordsBlockchain2020} did an extensive investigation into how
blockchain could improve transparency and trust in land ownership issues in
India. But their proposed solution was not implemented. Rather it functioned as
case study in what might have been rather than an empirical study of what is.
But there are some empirical results, if only few and far between. One group of
researches demonstrated through the development of a supply chain sourcing
system that traceability and transparency can be improved with blockchain
solutions \parencite{luAdaptableBlockchainBasedSystems2017}. 

But simply because something is possible doesn't mean that it will succeed.
While blockchain offers tremendous advantages due to the underlying technology,
technology adaption includes a socio-behavioral component.
\Textcite{kristofferfranciscoSupplyChainHas2018} note that adoption is slow and
real-world empirical results are rare precisely due to these socio-behavioral
components.

A study by \textcite{alazabBlockchainTechnologySupply2020} used survey data to
investigate the relationship between adoption and various variables. The main
findings are the trust, expressed both as inter-organizational trust and trust
in the technology impacts behavioral choices in blockchain adoption. Additional
work by \parencite{fossowambaDynamicsBlockchainAdoption2020} analyzed survey results of
senior IT leadership and determined that while leaders having differing opinions
about blockchain adoption, the network effect is paramount in overcoming
resistance to adopting blockchain. That is, pressure from trading partners to
adopting blockchain is strongly correlated to blockchain adoption in the supply chain
space.   

\Textcite{wongTimeSeizeDigital2020} used survey data and neural-network analysis
to examine blockchain adoption data and concluded that there are four major
considerations for managers in adopting blockchain: competitive pressures, the
complexity of the solution, the cost of the solution, and the relative
advantage. The researchers determined that senior leadership support, market
pressures, and regulatory support had minimal impact on blockchain adoption.
This is interesting in that it suggestions senior leadership's vision has
limited impact, at least for the participants in this particular study.
Speculatively, this maybe because technology leadership includes a significant
component of deriving authority through expertise, which senior leadership may
be seen as lacking. Alternatively, they may not adequately understand the
benefits of the technology for their company or industry.

Additional empirical research by \textcite{liuBehavioralTechnicalPerspectives2020}
corroborates the above findings. Looking at over 8 months of data collected
from 200 senior leaders in green supply chain management, the researchers
demonstrated that leadership lacking a vision for green supply chain management
will likely lack the wherewithal to further a blockchain strategy. The study
further demonsrated that the theorized positive relationship between green
supply chain management and performance is generally realized in developing
markets.

Multiple real-world case studies show that the theoretical benefits of
blockchain in SCM are frequently realized, from fish markets in Thailand
\parencite{tsolakisSupplyNetworkDesign2020} to the airport industry
\parencite{divaioBlockchainTechnologySupply2020} to many others.

In all of these studies, researchers are looking at actual adoption, the
attitudes of the leadership teams responsible for the SCM system development and
implementation and the perceptions of senior leadership, including their initial
and post-adoption support. A common theme arises that while blockchain is fairly
well understood and accepted in financial circles, in supply chain management
there remain numerous barriers to adoption
\cite{kouhizadehBlockchainTechnologySustainable2021,upadhyayDemystifyingBlockchainCritical2020}.

\section{Visionary Leadership}
\label{sec:org2b75779}

There is much that is happening in the blockchain world. Blockchain
cryptocurrency based funds are raising fund managers to new heights, with 11 of
the Forbes ``30 Under 30'' list for 2020 including fund managers focused on
cryptocurrency \parencite{castillo11BitcoinBlockchain2020}. But mainstream business
media is recognizing that the promises of blockchain have not yet matched the
realities. Issues of how to scale systems, how to mange costs (blockchain is
very energy intensive), and how to ensure security remain largely unsolved. But
leaders across industry verticals are working diligently to address these
issues \parencite{friedrichCouncilPostWhy}. 

\section{Areas of Further Research}
\label{sec:org827f273}

The research has shown that the promise of the technology can be born out. At
issue is the underlying psychological pressure to move leadership from
appreciating future promise of a technology to committing resources to realizing
those benefits. Thus one key area of research need is what are the specific
psychological and behavioral drivers that must be present to move leadership
to actively pursue an emerging technology as a leader in the field rather than
waiting for the masses to move first? 

A second key point of interest is precisely how transparent should sustainable
supply chain management systems be? The ideal public blockchain framework, where
anyone can read anything that is in the ledger is clearly not feasible as
details of contracts between suppliers and customers need to be maintained as
confidential. However, the real public benefit of this technology is to allow
consumers to see the critical behaviors of the entire supply chain so as to be
able to adjudicate if the supply chain's final product is worthy of their
support. For this technology to be transformative, much that is currently not
public needs to be made available. But the question of what information is
sufficient to that purpose and what information is best kept confidential (and
how that confidentiality is maintained) remains an intriguing area of research.

Finally, there is the simple issue of interoperability. Right now blockchain is
a generic term for a host of competing protocols, any one of which can be the
basis for a larger SCM standard. Right now, research into the costs and benefits of
different protocols to support a blockchain SCM system are underway. The ability
to have a unifying standard is necessary to allow small companies to ``join the
party'' as it where. Until that happens, each solution will be limited to the
application for which it is designed. So research into what features are
essential for a unified blockchain standard for supply chain management is
absolutely necessary. 







\printbibliography
\end{document}
