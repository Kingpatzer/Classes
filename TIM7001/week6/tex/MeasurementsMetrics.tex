\begin{frame}
  \frametitle{Measurements and Metrics}
  \framesubtitle{}
  \begin{itemize}
    \item Measure Traction Against Roadmap

          \note[item]{\scriptsize{The roadmap is the plan for success. However, it is, in an agile delivery pipeline, an indication of intention, not the plan to be followed. The roadmap will be updated each iteration based on what is learned during that iteration. Success factors defined up-front as business value to be delivered remains the constant. The timeline to deliver that value is assumed to be variable based on what is learned during development and implementation. Standard agile metrics for progression and value delivery are key metrics to showing progress.}}

    \item Measure Training and Adoption

          \note[item]{\scriptsize{As psychological factors are incredibly important to new technology deployment success, tracking training for leaders and employees alike is vitally important. This training must not be simply task based, as stated before. Rather, the training for leaders must be on how, specifically, to lead an agile program and provide for an environment of psychological safety. For employees, the training must be focused on how to think and behave in a way that fosters innovative solutions and learning.}}

    \item Measure Adherence to Process

          \note[item]{\scriptsize A key component of successful agile delivery is adherence to the agile process best practices. Therefore, measuring process adherence is more important than specific deliverables. So, for example, establishing metrics that demonstrate best Agile practices are followed (work meeting definition of ready/done, Work-in-progress adherence, pull-queues rather than push-queues, adherence to cadence, and so forth) will allow leadership to ensure that the organization is functioning as a learning organization. }

  \end{itemize}
\end{frame}
