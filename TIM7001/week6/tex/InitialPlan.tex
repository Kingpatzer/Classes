\begin{frame}
  \frametitle{Initial Plan}
  \framesubtitle{Initial Steps}

  \begin{enumerate}
    \item Identify Visionary, communicate role to organization
\note[item]{\tiny{As noted, a visionary leader is a proven success factor. Any major new technology implementation is a significant change for the organization. As such best practices of change management should be followed. This starts with communicating the need to change widely through the organization \parencite{kotterLeadingChangeNew2012}.}}

    \item Articulate reason for change, communicate widely
    \item Develop change plan and communicate widely

          \note[item]{\tiny{As the vision is being communicated, identify change champions, those who will willingly help support the change. Empower them to help lead the change in the organization.}}

    \item Identify change champions and build initial team
    \item Identify and engage consulting partner from top ERP consulting firms

\note[item]{\tiny{Engaging an experienced expert team to help direct the product selection and implementation is an essential element of success for complex projects. ``Few organizations can implement ERP successfully'' alone \parencite{tsaiEmpiricalInvestigationImpacts2011}.}}

    \item Update existing IT systems to modern standards, including cloud

\note[item]{\tiny{As we've already noted updating existing systems is a critical success factor and best practice.}}

    \item AMES (Agile Method for ERP Selection)
    \item Begin Agile Development Cycle
\note[item]{\tiny{Finally, utilize an agile methodology for both ERP selection and development. Agile methods rely on rapid iteration to ``fail fast,'' enhance organizational learning, minimize waste, and extract value from an IT system investment early \parencite{juell-skielseAMESAgileMethod2012, nagpalComparativeStudyERP2015, luUnderstandingLinkInformation2011}.}}
  \end{enumerate}
\end{frame}

