\begin{frame}
  \frametitle{Psychological Safety}
  \framesubtitle{An Under-Appreciated Necessity}
  \begin{itemize}
    \item Psychological Safety Speaks to Risk of Taking Personal Risk

          \note[item]{\scriptsize{Discovering how to succeed often involves the risk of failure. For workers tasked with learning and applying new technology, the risk for failure at any particular task is high.  }}

    \item A Lack of Psychological Safety Increases Anxiety

          \note[item]{\scriptsize{Being able to trust in the organization, the technology, one's ability to innovate, and other factors can increases anxiety. These all relate to psychological safety \parencite{edmondsonPsychologicalSafetyHistory2014}.}}

    \item Personal and Organizational Learning Increase With Psychological Safety

          \note[item]{\scriptsize{Learning is a key factor to being able to utilize new technology effectively. Psychological safety increases organizational and individual learning capacity (ibid). }}

    \item Psychological Safety is Driven by Leadership Behavior

          \note[item]{\scriptsize{Leadership behaviors have been shown by researchers to increase psychological safety (ibid). Therefore an important part of the implementation plan needs to be training for leaders on how to both champion learning and risk taking in the discovery process, as well as understanding how to react to inevitable missteps that happen with any new technology roll-out.}}

    \item Therefore: Leadership Training is a Key First Step




  \end{itemize}
\end{frame}
