\begin{frame}
  \frametitle{Innovation}
  \framesubtitle{Innovation Today}
  \begin{itemize}
    \item<1-> \alert { Innovation } intrinsically linked to technology

          \note[item]<1-> {\scriptsize{The link to technology for the term innovation happened in the 19th century.}}

    \item<2-> Has strong associations with economic and business success
    \item<3-> Can occur in multiple domains within business contexts
          \begin{itemize}
            \item<4-> Products
            \item<4-> Services
            \item<4-> Processes
            \item<4-> Marketing
            \item<4-> Business Model
          \end{itemize}

          \note[item]<4-> {\scriptsize{Products can involve innovation in two ways: new technologies or new technology applications. The same is true of services. This breakdown comes from \textcite{sbdcInnovationSmallBusiness2020}.}}
          \note[item]<4-> {\scriptsize{Process, Marketing, and Business Model are often lumped together as a single entity  parencite:pisanoYouNeedInnovation2015. But it seems more appropriate to segregate them. For example, different communication strategies can work with the same revenue model under different societal contexts.}}

    \item<5-> Applying creativity to solve problems \parencite{godinInnovationContestedIdea2015}

          \note[item]<5-> {\scriptsize{Innovation can be applied to many realms, as the SBDC notes, but according to Godin, Schumpter set the current use of the term is nearly exclusively reserved for employing technology in new ways or inventing new technology.}}
  \end{itemize}
\end{frame}
