\section{Actor-Netowrk Theory and Blockchain Adoption}

Actor-Network Theory can help with explaining the barriers to adoption for blockchain technology for supply-chain management. ANT was recently used to examine the prevalence of splits in public blockchain systems \parencite{islamWhyBlockchainsSplit2019}. However, no recent papers explicitly utilize this theory in relationship with supply-chain adoption.  It is still possible to utilize the lens of Actor-Network theory when exploring research on the topic.


\textcite{kouhizadehBlockchainTechnologySustainable2021} utilizes force-field theory to examine the challenges of blockchain adoption. However, this paper can easily be interpreted using ANT theory. The primary concept of ANT is that it accepts that knowledge is a product of a relational network between and everything that has some causal relationship to the production of an idea. In this paper, the researchers note that resistance forces ``stem from a variety of internal and external factors at different individual and broader organizational levels.'' These factors are precisely equivalent to ``actors'' in ANT, and their inter-relationship provides an basis for observing the sociological forces that prevent technological adoption.

\textcite{kouhizadehBlockchainTechnologySustainable2021} identifies three types of barriers, technological, organizational, and environmental. The technological barriers are not relevant for ANT analysis, but the others are. The most important ANT level criteria that are called out in this paper from an organizational perspective are a lacking of managerial commitment, organizational policies, expertise, and cultural agility. From an environmental standpoint, the paper calls out there is a lack of customer awareness, difficulty in collaboration, fear of information disclosure,  lack of government policies, market uncertainty, lack of external stakeholder involvement, and a general lack of external incentives. The researchers show how these various factors, again ``actors'' in ANT parlance, create conditions that stymie adoption efforts. Most importantly, these forces create a shared perspective that blockchain adoption is inherently difficult because so much change is required. While change is certainly a factor, the paper notes that it is the shared perceptions that drive inaction in adoption. Under an ANT lens, this is the network giving rise to a belief that drives behavior in the actors.

A second paper, by \parencite{kristofferfranciscoSupplyChainHas2018} examines the lack of adoption through a model that identifies similar factors. Again these factors can be seen through the ANT lens. The key factors identified are performance expectations, effort expectations, social influence, trust, and social influence. Through an ANT informed lens, this study explicitly class out that it is the socially constructed perceptions that arise within the network of actors that drives behavior. This is the exact claim of an ANT analysis. Unfortunately, this study doesn't break the large factors down to something specific enough to be considered an individual actor under an ANT analysis. Instead each factor is looked at as a broad category of individual perceptions. The paper specifically calls out that technological adoption and success is dependent on the relationship between multiple parties and their shared perceptions about the value and risk in adoption of the technology.

\textcite{alazabBlockchainTechnologySupply2020} performed an empirical analysis on reasons for non-adoption of blockchain. Their survey questions call out similar factors as discussed in the prior two papers. Again the network effect of shared beliefs is shown to give rise to collective beliefs and behaviors that limit adoption. This study provides for a superior ANT perspective from the previous one in that the specific assumptions (again, assumptions are themselves actors under ANT) behind expectations are explored. For example, important actors identified are assumptions about how challenging blockchain is to setup, how challenging it is to use, and how much effort it will take to operate. Additionally the authors break down social influence into specific network relationships, making it much more susceptible to being examined through an ANT framework.  The authors note the role that prior beliefs and assumptions of friends, family, and peers have on the establishment of beliefs which will lead to technology adoption or lack thereof. The authors note that presenting information, such as the success of companies like Starbucks have had in massively increasing supply-chain efficiency with little negative impacts and relatively low overall effort is not sufficient to overcome the influence of the impact of those other actors in the network.

\textcite{papathanasiouNonApplicationBlockchain2020} provides for a very unique look at the non-adoption of blockchain in the Greek shipping industry. This is significant because the shipping industry itself has seen significant adoption, with Maersk, the world's largest container shipping company, being an early leader in driving wide-spread blockchain adoption for supply-chain management. This study was conducted as a series of interview workshops where free-form answers could be analyzed for their content. The interviews revealed that the Greek shipping companies want the efficiencies, consistency, and security that blockchain can provide them. Moreover, they have the profit-margins to support projects to adopt new IT infrastructure to support blockchain adoption. Again, what was shown is that pre-existing beliefs, shared across a network of relationships functioned as ``actors'' in ANT terminology to drive incorrect conclusions and foment inaction. The workshops importantly called out that prior experiences with supply-chain software created beliefs and assumptions about how blockchain implementations would work that, regardless of their validity, were key drivers in the decision to not adopt blockchain. Indeed, one of the key defining characteristics of blockchain technology is data security is guaranteed, yet more than one participant specifically called out their belief that it was not secure as a reason to avoid implementing the technology. The power that these pre-existing assumptions have to operate within a network and create shared incorrect knowledge is particularly stark in this paper precisely because Maersk has already demonstrated to this very industry that those assumptions are incorrect. But, as ANT predicts, the network impact of actors is an incredibly powerful driver of social behavior, and adopting technology is a social behavior.

Finally, \textcite{fossowambaDynamicsBlockchainAdoption2020} performed an empirical analysis on supply chain adoption factors. They found that sharing knowledge, skills, best practices, and use cases both within firms and between firms has a strong positive impact on blockchain adoption. This is exactly the same result that ANT would predict. This study shows that actors have a strong positive effect when the actors are positively oriented towards adoption. As the prior studies all showed that incorrect assumptions and beliefs tend to re-enforce non-adoption, it is helpful to see that when a critical mass of actors favor adoption, that adoption becomes the behavior. This demonstrates that, again as ANT would predict, the constructed shared belief arises out of the interaction of actors in relationship. This means there must be a critical mass of individuals who have shared positive assumptions and knowledge favoring adoption before adoption will happen.

\section{Conclusion}

Blockchain technology has great potential to significantly improve supply-chain management to the benefit of suppliers, producers, and consumers. It has been adopted by some of the largest companies in the world and has proven benefits to efficiency, security, and transparency among other factors. Still, socially held assumptions and beliefs drive non-adoption across most business sectors. These beliefs can be understood to arise from the interactions of actors in relationship to one another. ANT is a framework for understanding how this sociological phenomenon gives rise to knowledge, both correct and incorrect. ANT has not been applied to the question of blockchain adoption within the supply chain specifically. Technology adoption is a sociological phenomenon driven by beliefs and assumptions, this suggests an opportunity to further investigate this relationship. Many current studies are specifically looking at reasons for non-adoption, and these studies can be understood using an ANT lens. There is room for further exploration and research on this specific phenomenon using ANT specific questions in the field.
