\section{Blockchain}

\subsection{Basics and History}

Blockchain is the name given to a secure, non-mutable, public ledger system that was invented in 2008 by an anonymous researcher who published under the pseudonym of Satoshi \Textcite{nakamotoBitcoinPeertoPeerElectronic2008}. The essential feature of this system, which underlies crypto-currencies such as Bitcoin and Eyhrtrum, is that the database, or chain, can only be added to when there is broad agreement across the network of users that the person adding to the data has the correct copy of the database. This is accomplished by utilizing cryptographic proofs. This system solves several complex problems, providing for an open-source, decentralized, linked ledger allowing multi-party transactions with theoretically perfect information sharing \parencite{kristofferfranciscoSupplyChainHas2018}.

Blockchain has numerous advantages if widely adopted, but it also has one serious concern, it requires an enormous amount of energy to perform a single transaction due to the cryptographic work that must be done to prove the validity of a transaction \parencite{kouhizadehBlockchainTechnologySustainable2021}.

\subsection{Supply Chain}

Since its inception the number of blockchain technologies has skyrocketed, and it shows great promise for significantly improving supply-chain efficiencies \parencite{wongTimeSeizeDigital2020}. However, adoption of the supply-chain framework has been much slower than predicted \parencite{alazabBlockchainTechnologySupply2020}.

Blockchain should be eagerly adopted by those with complex, global supply chains because the technology itself provides for a fault-tolerant, reliable, traceable, immutable ledger with no ability for any actor or even set of actors to engage in any degree of fraud. By reducing the need for intermediaries, blockchain can create major efficiency improvements in supply-chain management. Yet, in spite of these major advantages, investment in blockchain technology by major corporations actually is decreasing \parencite{kouhizadehBlockchainTechnologySustainable2021}. Still, it is showing very real success in managing sustainability issues, ensuring food safety, minimizing conterfeit products, and generally benig able to improve traceability and visibility to the entire supply chain. This visibility benefits both corporations and customers (ibid).

Blockchain technolology will allow for ``smart contracts'' between parties wherein the contract itself, embedded in a blockchain enabled shared ledger will provide demonstrated proof of the contracts fulfillment. This can be used to ensure suppliers respect human rights requirements, environmental sustainability concerns, sourcing requirements, and an innumerable collection of other possible contract requirements \parencite{tsolakisSupplyNetworkDesign2020}. The advantages are significant, so the slow rate of adoption presents researchers with a question: what are the challenges to widespread adoption?
