\section{Actor Network Theory}

Actor-network theory (ANT) originated in the late 1980s, incorporating ideas that had developed in the sociology of science over the preceding two decades \parencite{muniesaActorNetworkTheory2015}. \textcite{bijkerSocialConstructionTechnological2012} recalls that the framework for ANT came together initially as part of workshop presentation and was later incorporated into a chapter of a text by Latour. The original creators of ANT primarily worked out of the Centre de Sociologie de l’Innovation located within the Parisian engineering school and research center École des Mines de Paris \parencite{lawActorNetworkTheory2008}. Latour, one of the founding luminaries of ANT, explains ANT's role by talking about a bicycle accident \citeyear{latourWhenThingsStrike2000}. Latour explains that if a bicyclist falls over due to hitting a rock, a social scientist would ``have nothing to say.'' Latour contends that this isn't so, the bicycle, the practice of riding a bicycle, the road, even ``the geology of rocks, the physiology of wounds and so on'' either exist, are understood, or both, by the interaction of actors and the context within which actors understand and draw meaning from the world. As \textcite{muniesaActorNetworkTheory2015} contends, ANT is a ``a reaction to (and a dialogue with) two threads in the understanding of scientific inquiry: the French tradition of the epistemology of science and the British tradition of the sociology of scientific knowledge.''

The underlying essence of ANT is that actors, be they individuals, or large entities, come to understand their environment through a network of semiotic relationships that arise out of social context but which also create that social context \parencite{latourReassemblingSocialIntroduction2005}. In his text \textit{Science in Action}, \textcite{latourScienceActionHow1987} engages in a deep dive explaining how ``reality'' is apprehended through a constructivist social function of language, networks, and, most importantly, a ``performative'' function of metrics, methods, tools, and technologies whose importance and meaning arise from context and use.
