\section{Advocating for the new}

\subsection{Technology}

Of the articles examined, none focused on implementing new technologies at all. This fact is not surprising, as the articles chosen were focused not on specific technological solutions but the social, cultural, and political drivers of innovation in industry. The one article that was technology-focused, the \textcite{nevoExploringRoleIT2020} research on IT-enablement of creative behaviors, was technology agnostic in terms of the specific types of tools used. Instead, it focused on whether and to what extent tool use for the different phases of creativity within the life-cycle of bringing an innovative idea to fruition existed.

\subsection{Ideas}

The studies broadly fall into a few categories concerning ideas: those focused on the internal socio-cultural phenomenon that may be applied broadly, those focused on environmental factors that impact innovation, and those focused on interpersonal factors that impact innovation. In the first category of socio-cultural phenomenon land the studies of \textcite{bagKeyResourcesIndustry2021r} and \textcite{matsuzakiInnovationUpgradingLocal2020}. For environmental factors, the studies of \textcite{matsuzakiInnovationUpgradingLocal2020,nevoExploringRoleIT2020,tortorellaOrganizationalLearningPaths2020, aliEmpiricalStudyExplore2019} and \textcite{xieLeadershipStyleInnovation2018} are prime examples. Finally, for the interpersonal factors, there are the studies of \textcite{zhaoCharismaticLeadershipMillennial2021,chukwuemekak.echebiriEmpiricalStudyIndividualLevel2020,ahmadImpactWorkplaceInformation2020} and \textcite{isadaEmpiricalStudyRegarding2017}.  Each of the studies crosses categorical boundaries at various points. A few confounding variables generally limit each study, most frequently that of limited geographic sampling for the research. This presence of potential confounds means that all of these results exist within a the specific socio-economic, ethnic, national, and cultural confines of the sample population. Further, many of the studies were limited to organizations of a specific size. Still the authors are universally quick to call for additional research to pursue the topics disucssed.

\subsection{Research and Evaluation}

As discussed, the authors are quick to call for additional research to follow up on the ideas explored. The one clear exception to this observation is the paper by \textcite{isadaEmpiricalStudyRegarding2017}. This paper, as noted in the annotated bibliography, suffers from a multitude of problems, and does not present a robust call for specific follow-up research, but rather a general observation that additional research could be valuable. In terms of evaluations, the papers largely do not ask for their own findings to be evaluated in new ways, but rather, call for new research to explore how the evaluation of their research can fit within a larger, more robust picture of the interplay of the multifaceted factors that impact innovation adoption within corporations.

\section{Beneficial and Detrimental Effects}

From the picture of a broader societal discussion, many of the studies touch on specific questions that relate to each other in interesting ways. For example, \textcite{aliEmpiricalStudyExplore2019} explores organizational capabilities related to operations, \textcite{ahmadImpactWorkplaceInformation2020} looked at information literacy in the CEO, \textcite{tortorellaOrganizationalLearningPaths2020} looked at organizational learning, and finally, \textcite{nevoExploringRoleIT2020} investigated IT-enablement over the innovation life-cycle. These studies have clear points of intersection that both suggest new areas of research (such as how does IT-enablement impact CEO information literacy) but also suggests broader practical applications for benefiting society. As we grow in understanding as to how organizational capabilities are impacted by organizational learning, for example, policies can be crafted which provide specific business facing incentives to improve those impacts.

\section{Technology from a Global Perspective}

None of the articles touch on global issues directly. Each of the articles was focused on specific nations, or even regions within nations in the case of \textcite{matsuzakiInnovationUpgradingLocal2020}. Some of the papers, such as \textcite{chukwuemekak.echebiriEmpiricalStudyIndividualLevel2020} were looking only a specific industry within a relatively small nation. As such there is not much research here that can be responsibly extended to a global context without significant caveats. This doesn't mean these articles don't have global import, but rather, that they can only be responsibly used for such an application in conjunction with other research that demonstrates that the effects seen translate across regional, national, and cultural boundaries.
