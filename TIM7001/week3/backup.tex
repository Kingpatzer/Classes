% Created 2021-01-21 Thu 11:22
% Intended LaTeX compiler: pdflatex
\documentclass{apa7}
\usepackage[utf8]{inputenc}
\usepackage[T1]{fontenc}
\usepackage{hanging}
\usepackage{setspace}
\usepackage{parskip}


%% packages to set language standards
%% so, basically, babel

\usepackage[american]{babel}

%% packages used to insert linkedin graphic on title page
%% also clickable hyperlinks

\usepackage{hyperref}
\usepackage{fontawesome}

%% packages used for apa style considerations
%% including biblatex settings

\usepackage{csquotes}
\usepackage[style=apa,sortcites=true,sorting=nyt,backend=biber]{biblatex}


\doublespacing
\begin{document}


\title{Annotated Bibliography: Innovation Management}
\shorttitle{ANNOTATED BIBLIOGRAPHY}

\author{David A. Wagle}
\affiliation{Northcentral University}

\leftheader{Wagle}


\authornote{
  \addORCIDlink{David A. Wagle}{0000-0001-8130-4900}
  \hspace*{1.69in} \href{https://www.linkedin/com/in/davidwagle}{\faLinkedinSquare \hspace*{2pt} https://linked.com/in/davidwagle}

  Correspondence concerning this article should be addressed to David A. Wagle, School of Business, North Central University, C/O PO Box 1997, Burnsville, MN, 55337.

 E-mail: \href{mailto://david.wagle@gmail.com}{David.Wagle@gmail.com}}

\maketitle

\hfill\begin{minipage}{\dimexpr\textwidth-1in}
\begin{hangparas}{.25in}{1}
Bag, S., Yadav, G., Dhamija, P., \& Kataria, K. K. (2021). Key resources for industry 4.0 adoption and its effect on sustainable production and circular economy: An empirical study. \textit{Journal of Cleaner Production}, 281, 125233. https://doi.org/10.1016/j.jclepro.2020.125233
\end{hangparas}
\xdef\tpd{\the\prevdepth}
\end{minipage}
\\
\\
Researchers collected data from South African manufacturing firms. Utilizing a five-point Likert scale questionnaire, the questions asked were about ``Industry 4.0'' adoption, green design, collaboration, and sustainable practices. The team looked at 11 hypothesis examining relationships between attitudes towards sustainability and I4.0 adoption, adoption practices, and collaborative relationships with other firms. All 11 hypothesis held in the final analysis. The research presents a credible look at the socio-psychological barriers to building sustainable I4.0 companies in a developing economy. The study's goal, to establish a theoretical model linking key resources for I4.0 adoption is partially met but is limited due to the fact that the researchers examined companies only from a single country. As the study itself demonstrates, psychological factors impact adoption, and thus the study may be limited by cultural factors that can not be identified utilizing only a single country for the sample. The study strongly supports my own experience that relationships, leadership, and other human factors are the strongest predictors of a company's ability to adopt new paradigms, but must be supported by systems designed for the new paradigm, including IT infrastructure.

\bigskip
\hfill\begin{minipage}{\dimexpr\textwidth-1in}
\begin{hangparas}{.25in}{1}
Matsuzaki, T., Shigeno, H., Ueki, Y., \& Tsuji, M. (2020). Innovation upgrading of local small and medium-sized enterprises and regional innovation policy: An empirical study. \textit{Industrial Marketing Management}, S0019850119307631. https://doi.org/10.1016/j.indmarman.2020.07.009
\end{hangparas}
\xdef\tpd{\the\prevdepth}
\end{minipage}
\\
\\
The paper examines two specific research questions related to small-to-medium enterprises (SME): what internal measures promote innovation, and why and how do local policies impact innovation. The paper notes that a methodological challenge is defining degrees of innovativeness. Along with a number of fairly typical measures for internal innovativeness, the paper uses a proxy of ratio of R\&D investment to sales. The study looks at SMEs in the \textit{Hanshin} area and considered local policies of that region's prefectural government, their manufacturing academy support, and the industrial cooperation council and local support systems. These systems include financial subsidies for SMEs and internet support systems. The results show that both internal R\&D and open innovation are important factors for SME innovation. The regional manufacturing academy and the Industrial cooperation council are shown to have significant effect on regional SME innovation. The authors note the result is different from referenced literature on the topic. This research is an interesting look at innovation in a specific cultural context. As much of the specific policies discussed are relatively unique to Chinese prefectural systems, it is not directly applicable in many other locations. However, analogous programs and policies with similar goals do exist. A cross-cultural examination of similar policy effects could be an interesting followup study to see if the effect of policies with similar goals are as strong. The study provides an interesting insight into Chinese economic policy and the impacts of those policies on innovation. The study has utility in showing that innovation is impacted by governmental policy and that policy goals and results with respect to SME innovation of such policies is not easily predictable.

\hfill\begin{minipage}{\dimexpr\textwidth-1in}
\begin{hangparas}{.25in}{1}
Chukwuemeka K. Echebiri. (2020). An Empirical Study into the Individual-Level Antecedents to Employee-Driven Innovation. \textit{Technology Innovation Management Review}, 6(6), 42–52. https://doi.org/10.22215/timreview/1367
\end{hangparas}
\xdef\tpd{\the\prevdepth}
\end{minipage}
\\
\\
315 employees in the Norwegian banking sector are surveyed over time to evaluate the impact of individual autonomy and self-leadership with innovation. The innovation being looked at is specifically considered as ``employee driven innovation'' (EDI) which for this study involves employees voluntarily taking on roles extrinsic to their defined job functions in order to solve some business related problem. The hypothesis tested are rooted in the concept of Self-determination theory from the field of psychology. The research used two surveys delivered 10 days apart. The first sought out measures examining self-leadership and the need for autonomy and the second sought out measures examining EDI and perceived job-autonomy. The survey questions were adopted from existing research tools whose validity and reliance are already at least partially established. The results show a positive association between the need for autonomy and self-leadership; a positive relationship between self-leadership and EDI; but did not show a positive relationship between the need for autonomy and EDI. Two additional hypothesis on mediation effects were tested but did not show positive results. The paper provides more evidence to the growing body of research that suggests individual initiative, vision, and drive are important for innovation in many contexts. This provides additional weight to the insight that managers must be aware of the need to provide both freedom and a psychologically safe space for innovation to occur. This research adds to a growing body of studies and is not particularly important on its own. The value here is that the study is well constructed and the paper is very readable and accessable to non-experts. This finding aligns with my own experiences as an Agile coach doing numerous business transformation projects: much of the most important innovation activities happen from the ground up when employees are given the freedom to be innovative.


\bigskip
\hfill\begin{minipage}{\dimexpr\textwidth-1in}
\begin{hangparas}{.25in}{1}
Ahmad, F., Widén, G., \& Huvila, I. (2020). The impact of workplace information literacy on organizational innovation: An empirical study. \textit{International Journal of Information Management}, 51, 102041. https://doi.org/10.1016/j.ijinfomgt.2019.102041
\end{hangparas}
\xdef\tpd{\the\prevdepth}
\end{minipage}
\\
\\
This study looks at SME innovation and examines the relationship that the CEO's level of information literacy impacts exploratory innovation (innovation through experimentation), exploitative innovation (innovation through refinement of institutional knowledge), and exploratory innovation (changes in technology, markets, policy, etc.). The research was conducted by surveying CEO's of SME's in knowledge intensive fields in Finalnd. The surveys were sent out to over 1400 CEO's whose firms matched the study criteria, and the response rate was 9\%. The survey consisted of multiple five-point Likert scale questions. The CEO's industry experience and tenure were used as control variables. The study methodology included testing for both internal consistency reliability and indicator reliability as well as both discriminate validity and convergent validity. A total of 5 hypothesis were tested. CEO information literacy was shown to positively impact exploratory and exploitative innovation. This result confirms prior research. The results are also suggestive of opportunity recognition being higher for CEO's with high information literacy. The study is important because it shows that information literacy is important for CEO's in SMEs who wish to be highly innovative. This means that for these corporations, general business acumen alone is not sufficient in senior leadership. The study notes a gap in research examining how senior leadership information literacy impacts employee information capacity. Information literacy therefore being shown to be an organizational concern for innovation. This study starts attempting to bridge the gap between innovation and leadership. There is strong awareness in the research arena that there exists a relationship, but the precise nature of that relationship is not well elucidated. By linking information literacy to leadership's capability to spur innovation, that gap is narrowed. As information literacy, per this study, involves a great deal about information flow, the question arises as to how IT structures allowing for information flow impact information literacy? For low information literacy CEO's, do those CEO's not have the tools they need, or are those tools not used appropriately?

\hfill\begin{minipage}{\dimexpr\textwidth-1in}
\begin{hangparas}{.25in}{1}
Nevo, S., Nevo, D., \& Pinsonneault, A. (2020). Exploring the Role of IT in the Front-End of Innovation: An Empirical Study of IT-Enabled Creative Behavior. \textit{Information and Organization}, 30(4). https://doi-org.proxy1.ncu.edu/10.1016/j.infoandorg.2020.100322
\end{hangparas}
\xdef\tpd{\the\prevdepth}
\end{minipage}
\\
\\
The researchers in this study deconstruct innovation into several component steps as they relate to using IT in producing innovation: IT-enabled idea generation, IT-enabled idea elaboration, IT-enabled idea championing, and IT-enabled creative behavior. Utilizing the meta-theory of componential theory of creativity, the researches develop a set of hypothesis to describe how the elements of CTC may relate to the IT-enabled aspects of innovation. The research was a survey completed by 337 employees from within the United States in various organizations with at least 10 employees that have undergone a new IT implementation recently. The researchers developed scales for the IT-enabled components based on prior research on innovative work behavior. Reliability and validity checks were performed and passed. Six of the proposed mechanisms of the model had a positive coefficient at a 0.001 significance level. Two additional mechanisms of the 10 proposed were significant at a 0.05 level or better. This paper is significant because most research on IT-enablement of creativity focus on idea generation tools. This paper demonstrates that IT-enablement of creative activities is important beyond the initial generation of ideas. Critically, as it relates to organizational leadership, the notion of championing of ideas is rarely a component of innovation research, and including it here provides an important basis for further exploring the link between the championing of ideas and innovation success. This study's focus on the role of IT in innovation as an enabler of innovation explores an interesting corner of the innovation research space which seems very lightly explored. It will be interesting to see if this study generates any follow-up research to determine linkages between IT-enablement and cultural norms with respect to innovation. For example, are the same types of IT-enabled innovation championing tools that work in the USA effective in other cultures which may be more or less egalitarian?



\bigskip
\hfill\begin{minipage}{\dimexpr\textwidth-1in}
  \begin{hangparas}{.25in}{1}
Zhao, C., Tian, G., Wen, Z., \& Gao, X. (2021). Charismatic leadership and millennial employee innovation performance relationship mediated by employees’ leadership, professional, and organizational identification. \textit{Social Behavior \& Personality: An International Journal}, 49(1), 1–11.
\end{hangparas}
\xdef\tpd{\the\prevdepth}
\end{minipage}
\\
\\
The researchers study the results of charismatic leadership on 361 Chinese millennial (born between 1984 and 2001) employees. The hypothesis tested are that charismatic leadership has a significant positive impact on innovation, that leadership, professional, and organizational identification (viz-a-viz social identity theory) mediates the relationship between charismatic leadership and innovation, and that these mediating effects will serialize in their impacts. The employees worked in a variety of industries throughout China and completed surveys over a 5-month period. Charismatic leadership was measured using the Conger-Kanungo Scale while innovation was measured using the six-item Innovative Behavior Scale in use since 1994. The results obtained showed that there is partial mediation among the various proposed pathways. Charismatic leadership seems to have some positive impact among millennial employees but the effect is not particularly strong.  While prior literature has shown that charismatic leadership has a positive impact on innovation, this is one of the first studies to look specifically at young workers, who's values and motivations may not be reflective of the overall employee pool. The researchers are careful to note that their methodology does not allow for causal inferences. The study is particularly valuable because it utilizes existing and proven measurement tools. This allows for similar studies to be performed in different cultural contexts. This study raises many interesting questions: do these results hold across cultural boundaries? Is there variation across industries? What about gender differences within the millennial population? What about demographic information about the leader in question, do those matter? This is a fascinating exploration of a very human element of innovation management.


\bigskip
\hfill\begin{minipage}{\dimexpr\textwidth-1in}
\begin{hangparas}{.25in}{1}
Ali, Z., Zwetsloot, I. M., \& Nada, N. (2019). An empirical study to explore the interplay of Managerial and Operational capabilities to infuse organizational innovation in SMEs. \textit{Procedia Computer Science}, 158, 260–269. https://doi-org.proxy1.ncu.edu/10.1016/j.procs.2019.09.050
\end{hangparas}
\xdef\tpd{\the\prevdepth}
\end{minipage}
\\
\\
The researchers categorize managerial capabilities as the intersection of managerial style, decision-making and people development. Organizational capabilities is defined by two dimensions: process management, and performance management. Organizational innovation is defined as the ability to introduce new processes, products, or ideas into the organization that drives strategic outcomes. After reviewing the existing literature, the researchers develop a seven-part questionnaire. Data was collected across SMEs from Pakistan, with a final tally of 210 valid responses. After testing the model of validity and reliability, the researchers utilized the data to test six hypothesis generated by the interaction pathways of their model. OF the possible interactions, only management style enabling organizational innovation and people development enabling organizational innovation were not supported. The authors note that the study is limited as SMEs in developing countries ``are required to show some peculiar behavior,'' and thus the results do not readily align with similar studies done in developed countries. Of particular interest, however, is how closely these results would align to other developing countries with different cultural expectations. My personal experience is that Pakistani culture is very diverse, and  Pakistani people in general have a tremendous work ethic taking great personal pride in their contributions. This study provides a framework for managers to focus on organizational attributes to enable organizational innovation. As such it has very prescriptive application should it prove generalizable for SMEs in any developing country. A key component of the model that may not generalize well across cultural boundaries is decision making.





\bigskip
\hfill\begin{minipage}{\dimexpr\textwidth-1in}
\begin{hangparas}{.25in}{1}
Xie, Y., Xue, W., Li, L., Wang, A., Chen, Y., Zheng, Q., Wang, Y., \& Li, X. (2018). Leadership style and innovation atmosphere in enterprises: An empirical study. \textit{Technological Forecasting and Social Change}, 135, 257–265. https://doi.org/10.1016/j.techfore.2018.05.017
\end{hangparas}
\xdef\tpd{\the\prevdepth}
\end{minipage}
\\
\\
Building upon existing literature on the value of a transformational leadership style on innovation, this paper proposes a study that examines the interplay between transactional leadership styles, transformational leadership styles, and the role of trust as they relate to innovation. The researchers propose that individual identification and trust mediate leadership style to provide for an innovative atmosphere. They hypothesis that under this model, transactional leadership styles will not necessarily negatively impact having an innovative atmosphere, rather the level of trust will be a key determinant. The study examined 317 respondents utilizing a questionairre developed from the existing literature. Test for reliability and validity were conducted. The results showed that while transformational leadership styles were shown to positively correlate to producing an innovative atmosphere, while the transactional leadership style was not shown to be significant. Importantly, trust was shown to positively correlate with an innovative atmosphere and a transactional style was not shown to negatively correlate with trust. This paper demonstrates an important caveat to the correct assertion that a transformational leadership style is better for innovation, and that caveat is: provided trust and individual identity are equally respected by both leaders. But, this research shows that trust and individual identity are strong mediators to leadership style and a leader of any style can be successful if they engender an environment where trust and individuals are upheld. This research strikes me as a critical step in understanding why some transactional leaders can often be successful when transactional leadership as a style does not itself seem to be conducive to innovation. It further helps establish some specific components of team and corporate culture that support innovation.


\bigskip
\hfill\begin{minipage}{\dimexpr\textwidth-1in}
\begin{hangparas}{.25in}{1}
Isada, F., \& Isada, Y. (2017). An Empirical Study Regarding Radical Innovation, Research and Development Management, and Leadership. \textit{Naše Gospodarstvo/Our Economy}, 63(2), 22–31. https://doi.org/10.1515/ngoe-2017-0009
\end{hangparas}
\xdef\tpd{\the\prevdepth}
\end{minipage}
\\
\\
The authors of this research seek to examine what management styles are appropriate for radical innovative change compared to incremental innovative change. The authors developed a questionnaire to explore how R\&D management functions and the resulting level and type of innovative changes observed. The questionnaire was provided to 100 individuals who were employed as R\&D managers and also active students in the researcher's business school. Multiple factors were extracted from the questions and regression analysis was performed against 5 hypothesis. The results of the study did not provide any robust conclusions as to how management and leadership styles impact types of innovation. This study has a number of problems. First, the Journal itself is intriguing. It is a Polish publication, but the editors are primarily from Slovenia, and no editors are from Poland. It has a publication frequency of 4 issues per year, and was first published in 2015, yet it is currently on ``Volume 66.''  The paper itself is written in an odd style, which may be translation issues, but is sparse on details and has a very limited literature review to frame the research. Overall, it does not appear to be a particularly high quality study or paper. I include it here in part as an object of curiosity, but also because a published paper with negative results is also such a rarity. While I have many questions about the academic value of the journal and this paper specifically, I am pleased to see negative results being published as the positive result bias of academic publications is a known issue that limits intellectual advancement across many fields.



\bigskip
\hfill\begin{minipage}{\dimexpr\textwidth-1in}
\begin{hangparas}{.25in}{1}

\end{hangparas}
\xdef\tpd{\the\prevdepth}
\end{minipage}
\\
\\


\end{document}
