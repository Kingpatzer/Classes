\section{Malware}

Malware is nearly as old as the internet itself. In 1988, the internet consisted of only a few 10s of thousands of hosts \parencite{zimmermannInternetHistoryTimeline2016}, the htttp protocol and the World Wide Web did not yet exist, and most users still were associated to academic institutions. On November 2, 1988, roughly 10\% of those hosts began to act strangely. Robert Morris, Jr., who is now a respected computer science professor at MIT but was then a graduate student, had either accidentally or on purpose released the first internet ``worm'' a program that replicated itself, then sought out new hosts to infect, spreading its way across the nascent network \parencite{zittrainProtectingInternetWrecking2008}.

The worm was able to spread easily because controls where weak. Indeed, only three controls existed as a practical matter. First, there were ethical norms to not harm other people's equipment. Second the systems were built and staffed by professionals tho tended to keep them patched and configured properly. Third, malware lacked a business model \parencite{zittrainProtectingInternetWrecking2008}. Today,as Zittrain notes, all of this is no longer the case. Most devices connected to the network are owned by average people with no formal training in system maintenance and management. Malware developers have found multiple routes to profitability from directly scamming people to engaging in corporate espionage for hire. Finally, black-hat hacking has developed a following where the ethics to not harm others equipment is not a primary motivator of behavior. For these reasons, organizations have become aware of the need to have tools and techniques and trained professionals ready to prevent and respond to malware attacks.

\subsection{Snort}

One tool that is used quite frequently by large organizations to address the concern of malware is Snort. Snort is an open-source intrusion detection system that is maintained by Cisco SYstems, Inc. As an open source project, anyone in the world can contribute to Snort development. What makes Snort so powerful is that it uses rules rather than signatures to detect intrusion attempts. A signature looks for the specific behavior of a particular piece of malware. A rule, by contrast, looks for characteristic behaviors of malware. This allows Snort to detect 0-day attacks that do not yet have signatures developed \parencite{ciscoSnortNetworkIntrusion2021}.

Because Snort can both log network packets as well as perform rules based analysis on the packets, it is also possible to connect machine learning systems to the Snort logs to enhance the detection of intrusion attempts on the network. Complex analytical tools such as n-gram analysis and clustering algorithms can combine with other methods to achieve detection rates of 90\% or better, even for new attacks \parencite{khammasPrefiltersIntransitMalware2019,arraEvaluationPredictionImplementation2021}.

Snort thus provides several capabilities to an organization. A company can use Snort to prevent intrusions in the first place for the vast majority of malware attacks. Secondly, the company can use Snort to generate an alert to a security team when a suspected attack is initiated successfully. This will allow IT security experts to begin work on remediation as early as possible. Finally, by logging network traffic and classifying that traffic, Snort allows forensic investigations after the fact to determine how an attack was initiated and proceeded. This provides the organization the ability to learn from each successful attack and to improve the organizations ability to both prevent and respond to similar threats in the future.

\subsection{Solarwinds Patch Manager}

Along with human engineering, poorly patched machines are one of the top ways attackers can obtain unauthorized access to a system \parencite{Top10Cybersecurity2021}. Poor patching practice is such an important attack vector the third and fourth most exploited vulnerabilities in 2020 where Apache Struts exploits where patches have been available for 15 and nine years respectively. Part of the difficulty of patch management lies in the fact that our networks are built on a heterogeneous collection of vendor products. Collecting and analyzing the information necessary to determine which patches are needed from multiple vendors is a complex and thankless task.

To address this problem several companies, including Solarwinds have introduced patch management tools that make the job easier \parencite{solarwindsPatchManagementSoftware2021}. This product allows an administrator to easily identify critical patches based on the actual environmental configuration from a dashboard. The necessary patches can be automatically located and downloaded from the various vendors securely. Patching can often be automated and scheduled based on the criticality of the vulnerability the patch addresses. The important difference compared to services like Windows Server Update is that Solarwinds can know what third-party software is being used on a system and can patch those vulnerabilities as well.

This system allows a network administrator to ensure that servers and workstations using Microsoft Windows as the base OS are patched consistently. Further, patch histories can be recorded, reports can be created, and the data can be integrated into other IT Service Management tools such as the rest of the Solarwinds products.
