\begin{frame}
  \frametitle{Changing Face of Supply Chain and Logistics}
  \framesubtitle{A Brief History}
  \begin{itemize}
    \item<1-> Two Definitions of Supply Chain:

\centering{
           \begin{tikzpicture}
            \node[mybox] (box) {%
              \begin{minipage}{.50\textwidth}
                \tiny{
                1. The processes from the initial raw materials to the ultimate consumption of the finished product linking across supplier-user companies;

              2. the functions within and outside a company that enable the value chain to make products and provide services to the customer \parencite{coxAPICSDictionary1995}.}
          \end{minipage}
        };
        \node[fancytitle,right=10pt] at (box.north west){\tiny{Cox's Definition}};
        \end{tikzpicture}

\tikzstyle{mybox} = [draw=blue, fill=green!20, very thick,
    rectangle, rounded corners, inner sep=10pt, inner ysep=20pt]
\tikzstyle{fancytitle} =[fill=blue, text=white, ellipse]
%
\begin{tikzpicture}[transform shape, rotate=10, baseline=-3.5cm]
\node [mybox] (box) {%
    \begin{minipage}[t!]{0.5\textwidth}
      \tiny{ [Supply Chain is] \ldots  all of those activities associated with moving goods from the raw materials stage through to the end user.This includes sourcing and procurement, production scheduling, order processing, inventory management, transportation, warehousing, and customer service. Importantly, it also \textbf{embodies the information systems} so necessary to monitor all of those activities \parencite{quinnWhatBuzz1997}.}
    \end{minipage}
    };
\node[fancytitle] at (box.north) {\tiny{Quinn's Definition}};
\end{tikzpicture}

}
          \note[item]<1-> {\scriptsize{}}

  \end{itemize}
\end{frame}
