\begin{frame}
  \frametitle{Data Breach Causes}
  \framesubtitle{How We Lose Control}
  \begin{itemize}
    \item<1-> There are a variety of factors that contribute to data loss
          \begin{itemize}

            \item <2-> Organizational Factors
                  \begin{itemize}
                    \item <2-> IT Budget
                    \item <2-> Security Culture
                    \item <2-> Vendor Selection Process
                    \item <2-> Employee training
                  \end{itemize}

            \item<3-> Business Process Exposure
                  \begin{itemize}
                    \item <3-> Number and Complexity of Systems
                    \item <3-> Networking Topology
                    \item <3-> Size of Enterprise
                  \end{itemize}

            \item <4->  Technological Level of Security
                  \begin{itemize}
                    \item<4-> Authorization and Authentication Protocols
                    \item<4-> Data Storage and Encryption Standards
                    \item <4-> Threat Detection Maturity
                  \end{itemize}

          \note[item]<1-> {\scriptsize{While this is by no means an exhaustive list, it is helpful to think of the potential threats to our data security as a multidimensional surface. Each ``face'' of that surface represents an additional potential attack vector. The total surface area is the ``attack surface'' that exposes us to risk which must be mitigated. The ability to address each area of the key areas in organizational, process and technology spaces combines to establish the over-all risk exposre of the organization. \\ The examples under each sub-heading above are not exhaustive, but are merely examples of the types of components that reside under each category \parencite{dolezelManagingSecurityRisk2019}. The overall threat model is the combined impacts of all the surfaces exposed to attack through one or more means. }}

          \end{itemize}
  \end{itemize}
\end{frame}
