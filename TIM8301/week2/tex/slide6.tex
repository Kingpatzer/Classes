\begin{frame}
  \frametitle{Thinking Differently}
  \framesubtitle{Acknowledging the Illusion of Control}
  \begin{itemize}
    \item<1-> When Security Focuses on Compliance, Complacency Ensues

          \note[item]<1-> {\scriptsize{It is a well-understood phenomenon of social psychology that when rules are instituted, the perception of responsibility for the outcome is transferred from the person engaged in the actions covered by the rules to the rule-makers. If a problem is seen, the person acting will assume their perception is flawed because otherwise the police writer would have covered the situation. A personal sense of ownership and accountability for outcomes is lost and the burden to act with full responsibility is shifted away from the actor \parencite{phamInformationSecurityPeople2017}.}}

    \item<2-> Parallels in Phsyical Safety

          \note[item]<2-> {\scriptsize{When physical safety research started, command and control attempts to direct human behavior were tried and failed repeatedly. Systems were tightly coupled and unpredictable with emerging behaviors that rule makers could not predict. Management teams focused on safety empowered employees not merely to react to changing conditions and situations but to be responsible for them. Systems were created that were highly redundant, highly flexible, and expertise was valued over compliance to process and procedure. A social culture of shared ownership, demarcated by a ``no-blame'' atmosphere where learning from errors was more important than punishing error makers resulted in significantly higher safety rates \parencite{zimmermannMovingHumanasproblemHumanassolution2019}. In his famous book on the topic, \textcite{dekkerSafetyDifferentlyHuman2014} characterized this shift as people moving from a ``problem to control'' and on to ``a solution to harness.''}}

  \end{itemize}
\end{frame}
