\begin{frame}
  \frametitle{An Already Solved Problem}
  \framesubtitle{Agile Security}
  \begin{itemize}
    \item<1-> Command-Control and Process/Policy Based Knowledge Work Considered Harmful

          \note[item]<1-> {\tiny{Knowledge workers first appeared in a work by \textcite{druckerManagementProfessionalEmployee1952}. While Drucker used it to mean people who create knowledge, it has come to encompass any employee who possesses a hgh level of professional knowledge and who engages in the creation, transfer, or practical use of knowledge as part of their role \parencite{surawskiWhoKnowledgeWorker2019}. Knowledge workers are effectively those employees who's tasks include complex communication skills and critical thinking. A definition that covers most very office worker and modern manufacturing machine operator in American industry today. Practical experience in multiple arenas has shown that command-control structures over such workers reduces their effectiveness, efficiency, engagement, morale, and the overall quality of outputs \parencite{lencioniFiveDysfunctionsTeam2002,marquetTurnShipTrue2013,abrashoffItYourShip2012}.}}

    \item<2-> Empowerment is a Solved Problem

          \note[item]<2-> {\tiny{Given that command and control structures don't work, what structures do? Experience in the industry shows that using principles of business agility and leadership, which effectively amount to pushing decisions to the lowest possible level, allow employees to make such decisions without fear, trusting in their competency, and creating a safe, learning organization focused on continuous improvement, knowledge worker engagement, efficiency, and quality all tend to increase \parencite{reiferHowGoodAre2002,digital.aiStateAgileSurvey2020}. Of course such methods still require discipline around execution and commitment to sound working principles, but the issue of being able to create a learning organization of empowered, engaged knowledge workers is a problem with a known working solution. }}


    \item<3-> Security Can Follow This Model. But Needs Leadership!

          \note[item]<3->{\tiny{Just as Dekker (2014) showed that physical security and safety can be better solved by engaging employees as partners in the solution rather than regulating them to the being the subjects of command and control, new theories about cybersecurity suggest that the best way to address the complex dynamic cyber-environments that give rise to security issues is to move away from command-control, regulation, and governance as the primary driveres of security and instead focus on the social engagement and empowerment of people to be the drivers of solutions \parencite{zimmermannMovingHumanasproblemHumanassolution2019}. To do so however, requires industry leadership at multiple levels. First and foremost, leaders at the industry level must make compliance with strict prescriptive acts not be a reason in and of itself to fail external audits. Second, internal policies and procedures must focus on empowering SMEs to find, address, and react to issues without fear of reprisal. Lastly, companies must focus on building a culture of shared social accountability and awareness for cybersecurity. }}

  \end{itemize}
\end{frame}
