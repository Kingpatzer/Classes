\begin{frame}
  \frametitle{Specific Risks to Be Addresseed}
  \framesubtitle{Risks}

  \begin{itemize}
    \item Every employee is accountable to always be learning how to do security better

            \note[item]{\scriptsize{ Every employee being accountable for security may take time to become normative. The risk is that in moving to this model, security issues may develop because employees are not fully aware of their own accountability.}}

    \item Security must be a cultural touchstone

            \note[item]{\scriptsize{Cultural chanCultural change is hard and takes time. Getting employees to fully embrace security as a cultural element will require a significant effort.}}


    \item Every employee must be empowered to do security better

            \note[item] {\scriptsize{Empowerment to make changes in how security is approached will require constant communication, transparency, and support, not to mention training for some many people.}}

    \item Policy must follow sound practice, not dictate it or constrain it.

            \note[item] {\scriptsize{Getting employees to think holistically about best practices takes time, which means that policy can't simply be disregarded. Further, some policies will have to remain enforced due to external regulatory requirements.}}


    \item Leadership must be about results not process

            \note[item] {\scriptsize{There is a risk that existing bonus and evaluation paradigms will not readily facilitate the desired behaviors, and could drive anti-patterns.}}



  \end{itemize}
          \note[item] {\scriptsize{Moving from human as issue to human as solution is a major cultural change. Such changes always carry risk \parencite{yeganehSalientCulturalTransformations2020}. While the risks and costs of cultural change must be managed, the need to move from a perspective of people needing to be controlled and managed to people being a company's most important security asset presents a real opportunity to move ahead of the security curve \parencite{zimmermannMovingHumanasproblemHumanassolution2019}.}}


\end{frame}
