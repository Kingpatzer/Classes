\section{Introduction}

When multiple parties are faced with opportunities to balance maximum shared outcomes with minimized personal harms, behavioral economics tells us that often sub-optimal behaviors will arise spontaneously \parencite{viscusiBehavioralPublicChoice2015}. These irrational behaviors appear to be  relatively universal to the human condition and are somewhat predictable \parencite{arielyPredictablyIrrationalRevised2010}. Such results are often the product of conflicting goals making sub-optimal choices individually rational with respect to risk management even though such choices are, at the group level, overtly irrational and even harmful. In cybersecurity, there exist a phenomenon where competing public and private interests complicate fighting cybercrime. Policing agencies want to work with victims and prosecute criminals; industry victims want to resolve their risk exposure and limit negative press; and researchers want to collect and analyze data in order to improve protections and enhance prosecutorial abilities \parencite{atapour-abarghoueiResolvingCybersecurityData2020}. While all parties wish to see cybercrime prevented and cybercriminals prosecuted, each party finds that other parties resist cooperation. Industry victims wish to keep their data and information of the attack secret, or at least understated, so as to avoid negative press. Law enforcement wishes to keep details of the attack and methods secret so as to preclude copycat incidents. Researchers seek to keep data and information private so as to enable access to reluctant industry members. The result is that while all parties agree that sharing data is necessary to further their common goals, no party is ready (or sometimes able) to share their data with the others.

\section{The Problem of Data}

For technology-based companies, data is effectively the basis of their product or service. As such companies are reluctant to share data with researcher. Healthcare sector companies, for example, specifically balk at sharing, citing concerns relating to ethics, data security, costs, and public relations \parencite{pisaniProtectingUserPrivacy2019}. It is not only in healthcare where such concerns limit data sharing, in the shipping industry, even while large companies such as Merck invest hundreds of millions into end-to-end blockchain supply chain management systems, smaller shipping companies refuse to invest in such technology due to a desire to refrain from sharing data \parencite{papathanasiouNonApplicationBlockchain2020}. The modern direction of cybersecurity threat detection is to utilize machine learning, particularly neural networks, in intrusion detection systems, to flag attacks. Modern AI driven IDS systems have quite high accuracy for positive identification of for some types of attacks, some exceeding 97\%, however, they come with a false positive rate of nearly 10\%, which greatly diminishes their overall utility. Still other types of attacks had a positive identification rate of only 56\%, which is quite low \parencite{larriva-novoEvaluationCybersecurityData2020}.

Cybersecurity needs advanced AI techniques to keep up with the exponentially expanding attack surfaces presented by smart technologies and the internet of things. But the ability to create a truly robust machine learning model is dependent upon the availability of detailed data from confirmed attacks as well as data from confirmed non-attack events. Lacking such robust data sets, developers risk ``over-fitting'' their models, leading to high false positives \parencite{sarkerCybersecurityDataScience2020,larriva-novoEvaluationCybersecurityData2020}. This data should ideally include information about the company, metadata about the data itself, system configuration data, and other elements that most corporations consider highly confidential and proprietary. Therefore, a major hurdle to overcome in the cybersecurity space, both for practitioners, researchers, and companies is to develop a method for trusted sharing of proprietary data for cybersecurity use.

\section{Overview of Cybersecurity}

Cybersecurity is a broad topic that covers a broad range of activities. Today, the general standard for cybersecurity is the US Government's National Institute of Standards and Technology report CSWP 04162018, frequently referened to as the NIST framework \parencite{shackelfordBusinessCyberPeace2016}. This framework consists of a core set of activities, outcomes, and references for common critical infrastructure sectors. The material is organized to allow for communication of the activities and outcomes across organizations from the ``c-suite'' to the lowest level operators \parencite{nistFrameworkImprovingCritical2018}. The framework also provides a tiered method for evaluating the maturity of an organization with respect to the core activities.

The key framework functions are Indentify, Protect, Detect, Respond, and Recover. Each of these functions has subcategories to outline the scope of the function.

\subsection{Identify}

Under the Identify function are the categories of Asset Management, Business Environment, Governance, Risk Assessment, Risk Management Strategy, and Supply Chain Risk Management.

Within the communications and networking sphere, these strategies align with IT Service Management frameworks such as ITIL and COBIT \parencite{deborasuryawanInformationTechnologyService2018} with the exception of supply chain risk management. As these ITSM frameworks are frequently employed in larger communications and networking firms, addressing the NIST requirements for security in the networking and communications space for this function should focus primarily on ensuring that supply chain risk management is on par with the other categories. The emerging role of blockchain in supply chain management systems will do a great deal to facilitate supply chain risk management for all firms \parencite{minBlockchainTechnologyEnhancing2019,kshetriBlockchainRolesMeeting2018}.

\subsection{Protect}

Within the Protect function are the categories of Identity Management and Access Control, Awareness and Training, Data Security, Information Protection Processes and Procedures, Maintenance, and Proactive Technology.

This area is critical for communications and networking systems in particular because of the role that social engineering plays in gaining unauthorized access to data. Data can be encrypted both at rest and in transit, all security maintenance patches can be applied, and there can be robust intrusion detection systems in place, but if a user gives away their login credentials, then an attacker's job is basically complete. Indeed, while Awareness and Training is only one of the 23 total categories that are part of the NIST framework. However, social engineering attacks account for at least 50\% of all cybersecurity breaches \parencite{jeongImprovedUnderstandingHuman2019}. Technology can partially address this weakness, for example by requiring multi-factor authentication. But understanding of how human behavior and psychology can be exploited to breech security measures remains an area where attackers seem to continue to outpace security professionals.


\subsection{Detect}

The Detect function includes anomalies and events, security continuous monitoring, and detection processing.

In the computer networking space, these categories are more and more being addressed with artificial intelligence solutions \parencite{sarkerCybersecurityDataScience2020}. Detection of intrusion events, malware, network anomalies, and even software piracy using machine learning and deep learning have extremely high detection rates \parencite{chaudharyReviewVariousChallenges2020}. However, challenges still exist as false positive rates often require significant human intervention for appropriate actions to be taken. Of particular interest for the communications and networking sector is the discovery of techniques that can successfully detect and act against attacks happening over encrypted data channels \parencite{garciaDistributedRealtimeSlowDoS2021}.

\subsection{Respond}

The framework categories captured under response include Response Planning, Communications, Analysis, Mitigation, and Improvements. Once an event has been detected, it is important to respond to it appropriately. This starts with planning. For many companies, incident response is covered by their disaster response plan and business continuity planning. Problematically, in the author's professional experience, many businesses in the communications and networking space of paper-only plans. They very rarely, if ever, test their fail-over capabilities to ensure the validity of their plan because of fears of causing service disruptions. An area worth serious focus is bringing enterprise risk management and business continuity teams into closer alignment for most industries \parencite{russoConnectingDisasterDots2019}.

Additionally, the culture of cybersecurity tends to be insular, and still largely values obscurity as a security practice. Thus, Communication is an area in need of serious focus. One recent study found that key obstacles in effective response are ensuring that employees are kept informed, the right people are seeing the right information, and different organizational functional areas collaborate with appropriate prioritization of actions \parencite{ioannouCybersecurityCultureComputer2019}.

\subsection{Recover}

Under recover, the NIST framework includes Recovery Planning, Improvements, and Communications. Similarly to response, cybersecurity culture means that communications are not always effective. Of critical concern is that bibliometric reviews of the available literature suggest that recovery planning is not an area of significant focus for researchers. In the healthcare industry, for example, a recent study found that only 3\% of the publications studied touched on recovery planning \parencite{jalaliHealthCareCybersecurity2019}. This, in spite of the fact that experts are in agreement that recovery planning is essential to effective  recovery of a cybersecurity incident.

\section{General Observations}

While cybersecurity receives a great deal of attention in the media and from researchers, it does not often receive the attention required from business leadership to successfully address the numerous problems that exist. Further, the rising rate of cybersecurity attacks across network and communication systems means that this lack of attention is bound to have serious consequences. In 2020, nearly 70\% of businesses in an Asia-Pacific survey suffered from a successful cyberattack, with 55\% of the attacks being considered ``serious'' or worse. Despite this reality, cybersecurity budgets remain largely unchanged \parencite{CyberattacksRiseStagnant2021}.

Cybersecurity is a top-10 concern of IT management, and has been for at least a decade \parencite{chrismaurerCybersecurityItWorse2021}. However those same surveys show that for senior leadership, less than 36\% consider cybersecurity a top concern in 2019. Only 52\% of organizations surveyed include cybersecurity representatives in business strategy planning.

A famous talk given by Richard Clark, the then special adviser to the President on Cybersecurity at the RSA Conference in 2002 mentioned that the average firm spent less than 0.0025\% of their corporate budget on cybersecurity, an amount that is frequently less than that spent on employee coffee \parencite{lemosSecurityGuruLet}. The results from the aforementioned Asia study demonstrates that the apathetic and indifferent attitude towards cybersecurity has not changed drastically over the years.

The recent events of the COVID-19 pandemic, which resulted in many companies moving to a remote-work paradigm should have highlighted the lack of security planning, budgets, and preparedness, which one would expect would drive senior leadership concern. However, the apathy towards spending money on cybersecurity remains nearly constant \parencite{CyberattacksRiseStagnant2021}. One plausible explanation for this phenomenon is the psychological phenomenon of psychological distance \parencite{bendellHowFearLooming2020}. The idea here is that because cybersecurity breaches are big, impersonal events, that are theoretical future catastrophes, and removed from leadership's primary areas of personal responsibility, they seem highly improbable and thus leadership does not feel the psychological pressure to plan for them.

This is an unfortunate state of affairs, but it maybe addressed on grounds that do rise to the level of C-Suite engagement. First, cybersecurity deterence, prevention, and resolution can be driven through data-sharing models that meet the ``intrinsic needs of business efficiency'' \parencite{wangDrivingCybersecurityPolicy2020}. Further, since the vast majority of successful cybersecurity breaches are initiated through human social engineering attacks, raising employee engagement with regard to the nature of these attacks to a firms' employee base can go a long way to securing the environment \parencite{alshaikhAwarenessInfluenceModel2021}. It should be noted that this is more than simply training on the nature of social engineering, which has been shown to be only moderately effective overall. Rather, it involves ensuring employees are psychologically attached to the goal of promoting and providing cybersecurity. While a challenge, it is not a challenge that requires massive budgets. But it does require engaged leadership.
