\section{Mitigation, Continuity, and Disasters}

The Colonial Pipeline shutdown of May, 2021, demonstrates the economic damage that can result from unmitigated cyber attacks. The shutdown was attributed to the impact of ransomware being successfully injected into the company's IT systems \parencite{sangerCyberattackForcesShutdown2021}. These types of attacks are typically the result of either successful human social engineering attempts or attacks on web-presence weaknesses \parencite{sultanSurveyRansomewareEvolution2018}.

Social engineering attacks are among the most successful of attacks for a variety of reasons and are growing in sophistication regularly \parencite{krombholzAdvancedSocialEngineering2015}. Given the sheer number of employees of Accenture, it presents the largest opportunity for mitigation. This is not unique to Accenture. Social engineering is far less studied than other attack vectors, and researchers know less about why and how some social engineering attacks work and others fail. For example, what behaviors ranging from ``passive, non-volitional non-compliance'' to ``volitional (but not malicious) non-compliance'' are most useful for social engineering exploiters \parencite{dalalOrganizationalScienceCybersecurity2021}?

As was noted previously, Accenture's focus on training and monitoring compliance against social engineering attacks such as phishing attacks is an important, as well as expensive, component of their overall cybersecurity framework. However, prevention is not the same as mitigation and ensuring of business continuity. Additional steps are necessary.

The first step to mitigating against social engineering attacks is the discipline of testing the effectiveness of training and placing those who fail the training under heavy restrictions. While it is indubitably expensive and time consuming for Accenture to continually test employees for their ability to avoid social engineering attacks, it pays off by allowing Accenture to reliably identify employees who don't take social engineering attacks seriously. These employees have their access rights restricted and are subject to remedial training and heavier managerial oversight.

The second step to mitigating against social engineering attacks lays with the 100\% cloud-based architecture that Accenture employs. Accenture can identify data and applications as either requiring multi-times and hour backup, hourly backup, multi-times a day backup, daily backup, or weekly backup. Cloud based solutions allows easy snapshots of critical data. If data is lost due to ransomware attacks or some other attack, then rolling back to a point where the data was not inaccessible is fairly easy to do \parencite{amazonawsAmazonEBSFast2021}. By using a global cloud architecture with multiple, frequently tested, independent restoration paths, Accenture can guarantee a high degree of business continuity and disaster recovery from attacks similar to ransomware social engineering attacks.

Cloud based digital asset management also allows for high degree of flexibility and protection.  Mobile assets can be tracked automatically with geolocation technologies, and users who travel to regions where threat levels are high can have their access restricted using geofencing \parencite{syndigatemedia10WaysAsset2020}.

Cloud-based access control models allow the easy layering of multiple access models ontop of one another. Role-based access controls can have context-based access control models and attribute-based access controls models sitting on top of them.

Solid cloud-based architecture featuring frequently audited, role-based, least privilege access control also helps with mitigation. Access control mechanism preserve the so-called CIA triangle (confidentiality, availability, and integrity) from external attacks \parencite{malikConventionalStateoftheArtIoT2020}.

Because access is limited to a strictly role-based privilege, no single account has the access necessary to truly disrupt any segment of Accenture business. Developers don't have write access to datastores, for example. And those who create data are only given write-access to the data rows they need to write data to based on their role. Further, access control can serve not just preventative functions, but can be used to detect, correct, and recovery from unwanted events as well \parencite{malikConventionalStateoftheArtIoT2020}.

As an example of context-aware access controls, in association with asset control systems, a user can have their access limited or removed if their mobile device is not located where the user is supposed to be \parencite{jihContextawareAccessControl}. This simply control provides significant protections against lost devices or devices that are hijacked via remote connections. But it also offers client protection features.

Attribute-based Access Controls systems can serve as a secondary role-based model that is applied with greater granularity. For example, a consultant with multiple clients may hold multiple roles. By comparing the consultant's primary role with the client data and the data objects for which access is being requested, differing levels of permission can be provided on a data-object by data-object basis \parencite{huAttributeBasedAccessControl2015}. This fine-grained control means that attackers have very little hope of gaining wide-spread access to critical data.

Governance is another key part of the mitigation, continuity, disaster recovery process. Accenture has a 24x7 organization known as ASOC, the Accenture Security Operations Center \parencite{accentureEmpoweringOurPeople2021}. This group has a broad focus to protect the security and integrity of Accenture's assets, from physical buildings and people to technology assets to data and products. All of Accenture's security training starts and ends with a simple commandment: if there is a security event, the first order of business is to call ASOC.

This simply step can be remarkably effective in mitigating damage of an on-going attack. ASOC has the power, for example, to stop all data read-and-write-jobs to client databases based on the enterprise ID of the person calling them. They also have connections to investigatory and legal bodies in every country in which Accenture operations. This allows them to take immediate steps to engage legal entities to help in protecting potential damage to client's data or services as well.

ASOC also deploys agents to all devices identified through asset management to proactively protect against data loss \parencite{accentureDataLossPrevention2021}. These agents monitor for behaviors that are identified as risky, and flags and blocks such transfers in real-time. This strongly mitigates against data loss from successful cyberattacks, and helps ensure business continuity, thus preventing the need for disaster recovery.

With respect to disaster recovery, as has been noted, Accenture is 100\% cloud-based for all business services. Every service is required to have a tested and verified disaster recovery plan, and Accenture uses multiple cloud providers to ensure that even major regional disasters that take out multiple provider datacenters allow for easy recovery of operations. As each enterprise tool is built and managed as it's own entity, Accenture has a disaster recovery model that is demonstrably sustainable and performant because of their cloud architecture \parencite{andradePerformabilityEvaluationCloudBased2019}.
