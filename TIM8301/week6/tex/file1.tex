\section{Mitigation Through Controls}

Accenture, PLC. (https://accenture.com), is a multinational consulting company. In 2020 Accenture reported \$44.33B US in revenues and employed nearly 550,000 people \parencite{accentureplcAccenture2020Annual2020}. Accenture is the largest information technology consulting firm in the world by sheer number of employees and number of nations in which operates. It is second to Tata Consultancy Services in terms of total revenues. Accenture operates in more than 120 countries world-wide and its client list includes nearly 80\% of the Global Fortune 500, more than 90 of the Fortune 100, and many G20 Governments and Government Agencies \parencite{wagleClientAnalysisAccenture2020,accentureplcAccenture2020Annual2020}.

In order to ensure the security of both client data and Accenture's own internal data, Accenture includes numerous cybersecurity controls. Accenture is fully compliant with ISO/IEC 27001:2013, the international standard for information security \parencite{accentureInformationSecurityClient2021} as well as the European Union's General Data Protection Regulation (GDPR).

ISO/IEC 24001:2013 includes 114 controls that are divided into 14 control sets as part of Annex A. Of these mitigating controls, four stand out as being of utmost importance for a company of Accenture's size and scope. These are A.8 Asset Management, A.9 Access Control, A.7 Human Resource Security, and A.6 Organization of Information Security. This is not to say that the remaining areas are not highly important. Rather these four deserve special consideration simply because of the complexities involved when dealing with a corporation with the size and scope of Accenture.

\subsection{A.8 Asset Management}

With more than half a billion employees alone, the number of laptops, cell phones, tablets and other personal devices Accenture must manage is staggering. Along with these personal devices, Accenture has an enormous corporate infrastructure. Accenture is 100\% cloud-based in terms of their corporate functions, but they still maintain some assets in datacenters for research and client support purposes \parencite{AccentureCloudFirst2020,accentureplcAccenture2020Annual2020}.

A fundamental principal of cybersecurtiy is that a company ``cannot defend what they do not know they have'' \parencite{centerforinternetsecurityCenterInternetSecurity2021}. Asset management includes not merely awareness of the devices, but of locations, versions, licenses and lifecycles. Large-scale asset management is a complex process that requires advanced techniques to perform well \parencite{halimaibrahimkureAssetsFocusRisk2019}. To this end Accenture uses a complex system of in-house developed tools, commercial products, and advanced AI techniques to continually monitor and poll their assets for cybersecurity purposes.

This is one of the more expensive areas of cybersecurity management for Accenture. However, the cost is justified based purely on the importance of being able to ensure access points to Accenture's network, data, and client resources are known and properly monitored. Deployment of these tools is relatively straightforward as they are ``baked'' into the standard system images that are used for physical, virtual, and cloud devices. Once built into the deployment images, these tools are updated and configured automatically using remote management tool suites.

\subsection{A.9 Access Control}

Accenture follows the principal of least privilege to ensure that data, systems, and tools are only accessible to those who have a need to access them, and that the users' associated level of access grants them only the privileges they need to perform their role \parencite{centerforinternetsecurityCenterInternetSecurity2021,jillepalliHardeningClientSideGuide2017}.

This control is highly important for an organization that is largely cloud-based and spans the entire globe. It is unrealistic to think that half-a-million employees would all exercise proper judgement at all times with regard to application usage, data usage, and client confidentiality. Therefore, using access controls to ensure that only proper access happens is essential.

This is a relatively low-cost solution in practice as well. Accenture uses a single sign on system. This technology means that a user's access rights are stored as part of the access credentialing system. Routine audits and manager and director level access request approvals creates a multi-level set of checks to ensure proper access is maintained. Research has shown that collective, social, work-group level focus on security enhances overall security so there is some evidence that this type of structure is effective \parencite{chulwooyooCybersecurityTeamSport2020}.

\subsection{A.7 Human Resource Security}

Human Resource Security is easily the most costly and difficult to manage mitigating cybersecurity control. Accenture's process starts with pre-employment background security checks as part of the hiring process. Once hired, employees are required to attend literally dozens of cybersecurity training courses spread over the course of each year. This on-going training touches on every aspect of security, from how to configure home wifi networks to legal and regulatory demands. Passing in-depth quizes on these topics is a requirement for all employees, and failure to maintain training can be grounds for termination.

Additionally, employees behavior is monitored in a variety of ways. For example, fack phishing emails are regularly sent to employees. Employees who fail to spot fake phishing attempts and react correctly will have their mailbox and accesses restricted for a full year, their managers will be notified of their compliance failure, and they will be required to attend signficiant remedial training.

The cost-benefit analysis of cybersecurity training is, however, cost-effective as so many cybersecurity incidents start from employees failing to social engineering attacks \parencite{zhangCybersecurityAwarenessTraining2021}.

\subsection{A.18 Organization of Information Security}

Lastly, there is the organization of information security. The goal of Annex A.6 is ``to establish a management framework to initiate and control the implementation and operation of information security wtihin the organization'' \parencite{itgovernanceISO270012021}. For an organization the size of Accenture, this is incredibly important. It is entirely unreasonable to presume that disparate groups could coordinate the implementation and management of cybersecurity compliance and mitigation measures globally. Thus, a centralized body focused on this effort is needed.

Interestingly, this internal central organization, while complex to establish and operate, is actually very cost efficient. This is due to the fact that Accenture uses this internal team to learn and develop client consulting offerings which are revenue generating. Because cybersecurity is an important client offering, Accenture is thus able to leverage this team's internal activities against external clients.
