% Created 2020-11-14 Sat 16:39
% Intended LaTeX compiler: pdflatex
\documentclass[man]{apa7}
	 	 \shorttitle{JB SERVICE QUALITY}
\usepackage{hyperref}
\usepackage{fontawesome}
\usepackage{csquotes}
\usepackage{hhline}
\usepackage{colortbl}
\usepackage{arydshln}
\usepackage{caption}
\usepackage[utf8]{inputenc}
\usepackage[gen]{eurosym}
\usepackage[style=apa,sortcites=true,sorting=nyt,backend=biber]{biblatex}
\DeclareLanguageMapping{american}{american-apa}
\addbibresource{/home/david/Documents/School/References/bibliography.bib}
\affiliation{North Central University}
\leftheader{Wagle}
\authornote{
\addORCIDlink{David A. Wagle}{0000-0001-8130-4900}
\hspace*{1.69in} \href{https://www.linkedin/com/in/davidwagle}{\faLinkedinSquare \hspace*{2pt} https://linked.com/in/davidwagle}

Correspondence concerning this article should be addressed to David A. Wagle, School of Business, North Central University, C/O PO Box 1997, Burnsville, MN, 55337.
E-mail: \href{mailto://david.wagle@gmail.com}{David.Wagle@gmail.com}}
\abstract{A short examination of the characteristics of service quality, the dimensions most likely to be effective for Jelly Belly, and the role of management in the quality improvement process.}
\keywords{Quality, Quality Improvement, Baldridge Award}
\author{David A. Wagle}
\date{\today}
\title{Jelly Belly Service Quality Initiative}
\hypersetup{
 pdfauthor={David A. Wagle},
 pdftitle={Jelly Belly Service Quality Initiative},
 pdfkeywords={},
 pdfsubject={},
 pdfcreator={Emacs 28.0.50 (Org mode N/A)},
 pdflang={English}}
\begin{document}

\maketitle


\section{Introduction}
\label{sec:org27e5d68}

The Malcom Baldridge National Quality Awards is one of the premier recognitions of quality management in the world. Established by an Act of Congress in 1987, the award is a presidential honor for quality excellence. It is given annually in six different categories: non-profit, healthcare, education, small business, service companies, and manufacturing. As a food manufacturer, it is in this latter category that Jelly Belly would be eligible to compete. Any US-based organization can apply for the award. The award compares companies across multiple dimensions. These are:


\begin{APAitemize}
\item Leadership: How the organization is led and how it leads within its community.
\item Strategy: How well the organization plans and implements strategic goals.
\item Metrics and Analysis: Does the organization utilize data to manage key processes and performance.
\item Operations: The management of design, implementation, and improvement of processes.
\item Results: Rating the organization in terms of outcomes around customer satisfaction, financial performance, supply chain management, governance, social responsibiltiy, and human resource management.
\item Employees: How well are employes empowered, involved, trained, and engaged
\item Customers: How does the company due at retaining and pleasing customers.
\end{APAitemize}

More information about the particular award history and requirements can be found at the NIST.gov website, particularly the document outlining the methodology for award selection \parencite{ nationalinstituteofstandardsandtechnologyBaldrigePerformanceExcellence2015} .

\section{The Elements and Jelly Belly}
\label{sec:org8e9f907}
\subsection{Strategy}
\label{sec:org02ba4a1}

Jelly Belly's marketing strategy and their position as a leading name gives them ample room to experiment with strategic decisions. The commitment to growing their business through having programs designed to handle the smallest candy stores to large national relatilers demonstrates Jelly Belly's ability to be strategically agile. From a five-forces model, Jelly Belly's strategy is sound. Entering the confection business at scale takes massive investment, so threat of entry is low. Buyers at all levels are ultimately retailers who have limited bargaining power when dealing with the market leader. Suppliers want and need the certainty and consistency that dealing with an established company provides. There are few other jelly bean makers with the name recognition necessary to engage in real competitve rivalalry, and the confection business is not likely to see a price war anytime soon. Jelly Belly's only real strategic threat is substitute products, which exist in abundance. Still, from a strategy analysis perspective, Jelly Belly is strongly positioned to make innovative moves \parencite{ Porter1979, porterDynamicTheoryStrategy1991}.

\subsection{Leadership}
\label{sec:org3f67be2}

The first consideration is the involvement of leadership. Not only is leadership a critical component of the scoring system for the Baldridge award itself, but leadership performance is central to effective strategy, supply chain partnerships, and operational excellence \parencite{Tracy2015, wangTransformationalLeadershipPerformance2011,jaenDoesIntegrityMatter2020}. The involvement of leadership as active participants is key to successful transformational efforts, which is ultimately what quality improvement efforts are. Moreover, leadership should be  extremely interested in "getting it right," as there is ample empirical evidence that business excellence efforts pay handsome dividends when done well \parencite{paranitharank.pEmpiricalValidationIntegrated2017}. Luckily, Jelly Belly's leadership is intent on living up to the mission statement: "Jelly Belly Candy Company is dedicated to producing the highest quality confections, delivering superior customer service and creating a reliable and enjoyable product line to the consuming public. We seek to be a responsible corporate citizen and to ensure high quality and safety standards in our business. \parencite{jellybellyJellyBellyMission}".

\subsection{Metrics and Analysis}
\label{sec:org9e85e6a}

Jelly Belly is high on metrics, with both internal and external qualtiy measures available from the corporate information area of their website \parencite{ jellybellyOfficialSiteJelly}. Further, their processes are carefully watched at each step in the operations, as quality tests are performed on every batch at multiple stages in the process. \parencite{associationDiagnosticStatisticalManual2013}

\subsection{Operations}
\label{sec:org0cda1a5}

Jelly Belly's operations are in many places highly manual and lacking sophisticated, modern QA systems, such as atuomated sensors, or modern food processing systems to increase throughput such as high pressure ultrasound dehydration systems \parencite{zhangEffectsUltrahighPressure2020}. This is an area where Jelly Belly has the most room for improvements, but which must come at the impact of capital investment in new equipment.

\subsection{Results}
\label{sec:org396ee21}

Jelly Belly's results are exceptional at nearly every level. They have demonstrated financial stability in their industry. They are the number one selling producer in their category. They have phenominal name recognition. They have a strong, integrated supply chain which undergoes transparent audits for qualtiy as well as human trafficing and other social concerns.

\subsection{Employees}
\label{sec:org79d1b8f}

This is another area where Jelly Belly can do more. According to Glassdoor, Jelly Belly's employees are only moderately happy with their employment. There is a lot they like, but issues such as being punished for taking sick leave appear more often than one would expect from a company that truly values its employees.  Similar reviews appear on Indeed.com, where while the employees have good things to say about aspects of their work, they in general disapprove of the overall management style and culture \parencite{glassdoorJellyBellyCandy, indeed.comWorkingJellyBelly}. It would appear that focusing on HR processes and employee empowerment could go a long way to helping Jelly Belly improve.

\subsection{Customers}
\label{sec:orgfc293ed}

Jelly Belly customers have a high retention and satisfaction rate, as is demonstrated by their market position. Clearly, the company is doing something right.  They have a reasonably large social media presence and provide for a destination experience for their most devoted customers. Again, while there may be areas they can improve, this is not an area of significant concern.

\section{Conclussion}
\label{sec:org6e656b4}

Jelly Belly is likely not well positioned to compete for the Malcom Baldridge award at this point in time. While leadership is certainly saying the right things, and seems committed to quality in both their product outputs and in their partner relationships with both customers and suppliers, they have demonstrated little commitment to modernizing production operations or to employee empowerment and satisfaction. While the former can be addressed with some capital investments, 2020 has not been a good year for those producing and selling non-essential items. Jelly Belly may well be prepared to bounce back in the coming year, but presently engaging in large capital improvements to improve their operation systems ability to monitor quality is not likely worth the immediate risk, even though such investments may have a high long-term ROI.

More troubling, however, is the attitude of employees as demonstrated by employee review sites. This suggest that what is most needed for success by Jelly Belly to radically improve the quality of their system, from the holistic perspective of the Baldridge criterian is an organizational culture change. However, such change is remarkably difficult and has an amazingly low success rate. It is well understood by those working in organizational transformation that cultural change efforts have a failure rate of 70\% or more \parencite{burnesSuccessFailureOrganizational2011}. Moreover, success in such efforts is generally determined primarily by senior leadership recognizing a driving "why" behind the organizational change that is central to their own understanding of what is needed from a business perspective. External motivations, such as chasing an award, are not likely to provide that base level motivation \parencite{kotterLeadingChange2012}.

However, this does not mean that Jelly Belly should not take the criteria of the Baldridge awards and do nothing. By using these criteria to deeply analyze what Jelly Belly is doing well, and where it can most improve, Jelly Belly can focus their attention on the areas of improvement that will bring the most value to the organization.

\printbibliography
\end{document}