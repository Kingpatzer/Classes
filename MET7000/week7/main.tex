% Created 2020-12-03 Thu 11:08
% Intended LaTeX compiler: pdflatex
\documentclass[man]{apa7}
\usepackage[utf8]{inputenc}
\usepackage[T1]{fontenc}
\usepackage{graphicx}
\usepackage{grffile}
\usepackage{longtable}
\usepackage{wrapfig}
\usepackage{rotating}
\usepackage[normalem]{ulem}
\usepackage{amsmath}
\usepackage{textcomp}
\usepackage{amssymb}
\usepackage{capt-of}
\usepackage{hyperref}
\shorttitle{SUPPLY CHAIN MANAGEMENT}
\usepackage{hyperref}
\usepackage{fontawesome}
\usepackage{csquotes}
\usepackage{hhline}
\usepackage{colortbl}
\usepackage{arydshln}
\usepackage{caption}
\usepackage[utf8]{inputenc}
\usepackage[gen]{eurosym}
\usepackage[style=apa,sortcites=true,sorting=nyt,backend=biber]{biblatex}
\DeclareLanguageMapping{american}{american-apa}
\addbibresource{/home/david/Documents/School/References/bibliography.bib}
\affiliation{North Central University}
\leftheader{Wagle}
\authornote{
\addORCIDlink{David A. Wagle}{0000-0001-8130-4900}
\hspace*{1.69in} \href{https://www.linkedin/com/in/davidwagle}{\faLinkedinSquare \hspace*{2pt} https://linked.com/in/davidwagle}

Correspondence concerning this article should be addressed to David A. Wagle, School of Business, North Central University, C/O PO Box 1997, Burnsville, MN, 55337.
E-mail: \href{mailto://david.wagle@gmail.com}{David.Wagle@gmail.com}}
\abstract{This paper examines the differences between supply chain management visions of FedEx and Price Waterhouse Cooper (PWC). The separate visions of supply chain management, compared and contrasted with each other, lead to an evaluation of how supply chain management effects business operations. The role of supply chain management as a competitive advantage is considered, as well as how a company might initiate implementing supply chain management into operations management.}
\keywords{Supply Chain, Operations, FedEx, PWC}
\author{David A. Wagle}
\date{\today}
\title{Supply Chain Management: A Comparison of Two Market Leaders}
\hypersetup{
 pdfauthor={David A. Wagle},
 pdftitle={Supply Chain Management: A Comparison of Two Market Leaders},
 pdfkeywords={},
 pdfsubject={},
 pdfcreator={Emacs 28.0.50 (Org mode 9.4)}, 
 pdflang={English}}
\begin{document}

\maketitle


\section{Introduction}
\label{sec:org1059743}

The DoD has an enormous supply chain system, and at the start of the century, had enormous problems in running it effectively. A source of numerous studies, \textcite{haraburdaSupplyChainManagemetn2017} utilized the DoD to establish a demonstrably functioning supply chain maturity model. This model measured 23 functional components over six functional areas (Organization/Workforce, Performance/Metrics, Resources, SCM Processes, Technology, and Vision/Strategy) in order to determine the relative maturity of a supply chain. This model will form a background against which FedEx's and Price Waterhouse Cooper's (PWC) relative approaches to supply chain management can compare.

\subsection{Maturity}
\label{sec:org14d4681}

The maturity model provides for five levels of maturity, they are:

\begin{enumerate}
\item Level 1 -- Initial: At this level of maturity, inventory tracking exists. Inventory ties back to contracting and finance. The organization can manage assets to support demand, can manage the transportation, disposition, and acquisition of inventory. There will be requirements for inventory. Finally, there are processes for maintaining equipment and facilities.
\item Level 2 -- Managed SCM: There is functional integration between business functions such as finance and operations. This integration allows for deployment and allocation of organizational resources effectively and efficiently. Inventory balance includes financial considerations and impacts. Modern technology comes into play in order to improve operational performance. The company collects metrics and analyzes them. The workforce employs SCM specific task. The technology can provide timely information about inventory assets to workers who need that information. There is a continuous improvement program in place.
\item Level 3 -- Tailored SCM: Relationship management with suppliers is front and center of SCM considerations. The company employs balanced scorecards and benchmarking. Customer relationship management tools are utilized in SCM activities. The supply chain is integrated from suppliers to customers. Sourcing is strategic in nature. Performance-based logistics ensure optimized product availability, balancing costs with other strategic considerations.
\item Level 4 -- Tailored SCM: Enterprise integration ensure information exchange between business functions is free from friction and happens in a way to support strategic goals seamlessly. Strategic planning and execution is central to organizational direction and includes supply chain management considerations.
\item Level 5 -- Optimized SCM Integration: Operational excellence means that organizational performance results in continuous improvement to value delivery to customers and stakeholders. The organization can turn its focus to sustainability.
\end{enumerate}

\subsection{FedEx Supply Chain Management}
\label{sec:org7421a2c}

FedEx focuses on providing managed services to off-load aspects of supply chain management to them as a third party provider. These services include Enterprise housing \& management, critical inventory and service parts logistics, e-commerce fulfillment, reverse logistics \& returns, Return processing, and Repair \& refurbishment \parencite{fedexSupplyChainServices}.

FedEx's focus is on the functional, operational delivery of specific components of supply chain management as third-party services. This solution certainly can have benefit for companies that have little to no experience in supply chain management. FedEx specifically calls out the utilization of software tools to integrate and optimize supply chain steps. They also provide consulting services around compliance and regulatory requirements. They do customized tool implementations specifically focused on warehouse management and transportation management.

\subsection{Price Waterhouse Cooper Supply Chain Management}
\label{sec:org685fb38}

PWC provides for multi-year consultant-led transformation programs utilizing customized software tools from leading manufactures who are PWC partners. The case-studies they present to represent their work in the area of supply chain focus on implementing advanced end-to-end ERP/CRM tool chains, to include advanced AI components aimed at driving continuous improvement. PWC seeks to map out internal processes and align them to operational value streams, each with defined key performance indicators, in order to fully manage a mature, integrated supply chain \parencite[for example, see][]{pricewaterhousecooperHowAdvancedDigital}.


\section{Comparing and Contrasting The Two Visions}
\label{sec:orga572ede}

\subsection{Key Similarities}
\label{sec:org1124784}

Both FedEx and PWC recognize the importance of technology in ensuring efficient and effective supply chain management. A case-study by \textcite{landscheidtEvaluatingFactoryFuture2019} found that technology maturity forms a critical component of supply chain management. Likewise, they both recognize the changing needs of SCM systems in an ``Industry 4.0'' world, where on-demand ordering and rapid delivery cycles present unique challenges, not previously seen \parencite{aviles-sacotoGlanceIndustrySupply2019}.

Further, both clearly understand the need for integration of the supply chain and end-to-end visibility as a requirement for being able to optimize efficiently. This becomes even more important as AI-based optimization systems become more prevalent \parencite{praveenInventoryManagementCost2019}.



\subsection{Key Differences}
\label{sec:org8aafe52}

While FedEx does offer some enterprise-level software customizations, their focus is obviously not on organizational transformation. That is, they are not interested in taking on the challenge of developing a culture of continuous improvement within a company, process mapping, establishing corporate KPI's and so forth. These activities are central to reaching the upper levels of the maturity model previously discussed and have been shown to contribute to organizational performance \parencite{keawkuntiRelationshipSupplyChain2020}.

FedEx presents turn-key solutions to multiple activities that an organization may wish to outsource from the sub-arenas of supply chain management such as warehouse management, or order fulfillment. PWC does not present such turn-key solutions and instead focuses on building a plan with their client on how the client might meet their SCM goals. It is conceivable that with the information presented, PWC might suggest outsourcing particular functions to FedEx as part of a comprehensive strategic plan. The converse is not at all likely.


\section{Supply Chain Management's Impact On Operations}
\label{sec:org377c512}

At the lowest levels of maturity, supply chain management will have little impact on operations. However, at even moderate levels of maturity, impacts on operations will be great. When inventory tracking combines with procurement tracking, then just-in-time (JIT) deliveries for planned manufacturing runs become possible. This alone provides for a significant reduction in the cost of goods sold \parencite{cortes-comererJITMadeOrder1986}.

At the highest levels of maturity, supply chain management involves the inclusion of all relevant stakeholders, from the lowest level supplier in the supply chain through the end customer in operations decisions. Information from each of these levels will be presented to operations leadership regularly, and decision making for operations will become data driven by that information.

Further, as supply chain management maturity requires maturity in other arenas of the business, such as finance and customer management, and supply chain management is focused on fulfilling strategic goals, operations management must of necessity become more strategy focused for supply chain management to improve. For example, operations managers will need to consider impacts to the cost of quality for supply chain management changes in suppliers' operations. An example of this would be Apply engaging in audits of off-shore suppliers for workers' rights compliance. When Apple's suppliers make changes in their policies, it behooves Apple to notice and adjust how they ensure compliance in order to see the full benefit of the integrated supply chain.

It is fair to say that supply chain management, at the highest levels of maturity, drives operations more than operations drives supply chain management. At these levels, integration of the supply chain will drive nearly every operational decision. For example, decisions about if, where, and when to expand manufacturing would start from data generated by both SCM and CRM systems, processed by AI-based supply-chain management data analytic tools.

\section{Supply Chain Operations as Competitive Advantage}
\label{sec:orgb00ed0c}

It is well understood that JIT provides for competitive advantage \parencite{yangAchievingJustTime2021}. SCM of course, is necessary to provide for JIT capabilities. But there is additional key points to consider. The sharing of high quality information within a strong partner of suppliers is strongly associated with competitive advantage \parencite{keawkuntiRelationshipSupplyChain2020}. Recent research shows that firms with mature supply chain management systems create competitive advantages that are possible only because of the availability of high quality SCM data to be employed in big-data analytic tools \parencite{herdenExplainingCompetitiveAdvantage2020}.

The simplest way to understand SCM creating competitive advantage is to consider the degree to which SCM requires coordinated business strategies across every strata of interlinked strategic partnerships with suppliers, as well as integrated CRM systems providing insight to customers. Business strategy created with easy access to all relevant data is almost guaranteed to be superior to business strategy created with none of that information. SCM allows for business decisions to be made with much greater awareness of the likely outcomes.

\section{Implementing Supply Chain Management for the First Time}
\label{sec:org357cef9}

It may seem that with the foregoing discussion, that the resulting conclusion would be that PWC is providing a superior solution to SCM than FedEx. That is not the case. SCM provides for a client to access higher levels of maturity than FedEx for larger clients with more complex supply chains. But for a company that is without any meaningful supply chain management, any step that elevates the maturity level of the company is valuable. Small or otherwise immature companies are well-served to implement turn-key solutions to help them enjoy at least some level of SCM maturity with a low cost-to-entry. PWC's numerous case studies all speak to multi-year solutions, custom made to fit the client needs. Such projects are the only way to reach the highest levels of maturity specifically because of all the internal and external integration steps necessary.  But they are not a good way to start doing supply chain management for the first time.

Using resources like those provided by FedEx, smaller companies can get a handle on all of the items of Level 1 maturity, and many of the key considerations for Level 2. But equally important, they can achieve that with very little time investment. Because it is established that SCM provides for competitive advantage, a temporally bound thing, it is imperative that companies without adequate SCM address the lack quickly. Therefore, any company without SCM would be well-served to begin implementation by contracting with someone like FedEx who provides turn-key solutions while simultaneously ensuring that key operations personnel receive adequate training in supply chain management and continuous improvement.
\parencite{zhouMediatingRoleEmployee2014}
\parencite{Bauer,wittgensteinPhilosophicalInvestigations1973,winermanCriminalProfilingReality2004}

\parencite[testing][]{zhuBriefExposureMisinformation2012} 
\printbibliography
\end{document}
