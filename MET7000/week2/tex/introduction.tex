\subsection{Manufacturing Context}
\begin{frame}
  \frametitle{Introduction}
  \framesubtitle{Understanding Medical Manufacturing Context}
  \begin{itemize}
    \item<1-> Medical Manufacturing is Unique
      \note[item]<1-> {\scriptsize{Medical device manufacturing is a highly regulated industry that faces unique challenges not found in any other sphere. Device manufacturers are covered by a combination of ISO standards, Federal Regulations and Federal laws. Manufacturers must obtain FDA approvals at multiple steps along the product development, marketing and manufacturing timeline. Further, FDA requirements include tracking some devices through the life of the device with the end user \parencite{johnsonFDARegulationMedical2016}}}

    \item<2-> Medical Manufacturing is High Risk
      \note[item] <2-> {\scriptsize{Class II and Class III (devices which are non-disposable functional items rather than disposable equipment like syringes or tongue depressors which are Call I devices) Medical devices are held to the highest quality standards possible, and are continually monitored for malfunction over the life of the device. Any pattern of failure or malfunction can result in expensive recalls. A 15 year between 2002 and  2016 found 806 million devices were recalled \parencite{ghobadiApprovaladjustedRecallRates2019}. }}

    \item<3-> Medical anufacturing is International
      \note[item] <3-> {\scriptsize{Due to the expense involved in getting devices approved, ensuring high quality standards, tracking them, and the risk of recall, devices are manufactured for an international market. This subjects manufacturers to multiple standards administered both by international treaty and by international standards bodies like the ISO \parencite{schuhCompilationInternationalStandards2019}.}}

  \end{itemize}
\end{frame}
