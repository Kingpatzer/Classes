\subsection{General Strategy}
\begin{frame}
  \frametitle{Medical Device Development}
  \framesubtitle{Operationalizing Innovation}
  \begin{itemize}
    \item<1-> Product design is inherently linked to corporate strategy
      \note[item]<1-> {\scriptsize{Regardless of the industry, the design of a new product is linked to corporate strategic decisions about market participation, segmentation, differentiation, and concentration \parencite{Tracy2015}.}}
    \item<2-> Major strategic considerations
      \begin{itemize}
        \item<2-> Market Pull
        \item<2-> Technology Push
        \item<2-> Interfunctional View
      \end{itemize}
      \note[item]<2-> {\scriptsize{When deciding to bring a partoduct to market one can focus either on meeting market needs as expressed by customer inputs or the company can have internal technology that they decide to attempt to push to the maret. Lastly, companies can use an integrated approach that calls for a product to fit the market but also provide for a technology advantage \parencite[][p. 41]{schroederOperationsManagementContemporary2010}.}}


    \item<3-> Medical device strategy for ``Medical Device Co.'' follows the third option
      \note[item]<3-> {\scriptsize{The client company looks for opportunities where technological advancement can be married to a defined market need. This is common to the medical device manufacturing environment due to the lengthy and costly process of bringing new devices to market. This strategic choice is a product of necessity due to the high cost of bringing product to market in the medical device arena. }}

  \end{itemize}
\end{frame}
