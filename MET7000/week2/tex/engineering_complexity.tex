\subsection{Engineering Complexity}
\begin{frame}
  \frametitle{Medical Products Are Highly Complex}
  \framesubtitle{Presenting Unique Operational Risks}
  \begin{itemize}
    \item<1-> Medical devices are complex, including biochemical, physical, and computer engineering elements
      \note[item]<1-> {\scriptsize{Medical devices have components that are dependent upon every concievable type of production engineering. This means that operations management needs to be expert in the handling of nearly every conceivable type of production, from embedding software in microcontrollers to biochemical engineering. }}

    \item<2-> Medical device production therefore includes physical production as well as chemical and computer engineering

    \item<3-> Medical device liability requires the highest standards of quality and reliability
      \note[item]<3-> {\scriptsize{This means that operations management has to really focus on quality management in all areas. One arena where there is signifcant quality control issues but which is often not considered by those not familiar with the industry is the computer security requirements of medical devices. Because many devices are highly computerized, they are vulnerable to ``hacking'' or illicit access to the device. This can compromise both the health and privacy of patients. As a result, focusing on security is yet another unexpected priority for medical device manufacturers \parencite{hagerSecuringPrivateMedical2020}}}

  \end{itemize}
\end{frame}
