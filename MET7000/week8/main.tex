% Created 2020-12-06 Sun 13:37
% Intended LaTeX compiler: pdflatex
\documentclass[man]{apa7}
\usepackage[utf8]{inputenc}
\usepackage[T1]{fontenc}
\usepackage{graphicx}
\usepackage{grffile}
\usepackage{longtable}
\usepackage{wrapfig}
\usepackage{rotating}
\usepackage[normalem]{ulem}
\usepackage{amsmath}
\usepackage{textcomp}
\usepackage{amssymb}
\usepackage{capt-of}
\usepackage{hyperref}
\shorttitle{Operational Efficiency}
\usepackage{hyperref}
\usepackage{fontawesome}
\usepackage{csquotes}
\usepackage{hhline}
\usepackage{colortbl}
\usepackage{arydshln}
\usepackage{caption}
\usepackage[utf8]{inputenc}
\usepackage[gen]{eurosym}
\usepackage[style=apa,sortcites=true,sorting=nyt,backend=biber]{biblatex}
\DeclareLanguageMapping{american}{american-apa}
\addbibresource{/home/david/Documents/School/References/bibliography.bib}
\affiliation{North Central University}
\leftheader{Wagle}
\authornote{
\addORCIDlink{David A. Wagle}{0000-0001-8130-4900}
\hspace*{1.69in} \href{https://www.linkedin/com/in/davidwagle}{\faLinkedinSquare \hspace*{2pt} https://linked.com/in/davidwagle}

Correspondence concerning this article should be addressed to David A. Wagle, School of Business, North Central University, C/O PO Box 1997, Burnsville, MN, 55337.
E-mail: \href{mailto://david.wagle@gmail.com}{David.Wagle@gmail.com}}
\abstract{This paper is the final project for MET7000. The paper examines a fictional product launch to present one plausible case study of operations management. The paper attempts to synthesize various operational topics in order to demonstrate how one might think about operational management from the standpoint of a small company.}
\keywords{operations management, innovation, operations, case study}
\author{David A. Wagle}
\date{\today}
\title{An Examination of a New Product Launch for Operational Efficiency}
\hypersetup{
 pdfauthor={David A. Wagle},
 pdftitle={An Examination of a New Product Launch for Operational Efficiency},
 pdfkeywords={},
 pdfsubject={},
 pdfcreator={Emacs 28.0.50 (Org mode 9.4)}, 
 pdflang={English}}
\begin{document}

\maketitle


\section{Introduction}
\label{sec:org6783052}

\Textcite{tsayReviewProductionOperations2018} note that production and operations management includes considerations of the ``theory of the firm.'' This is discussion of how a corporation determines what actions it will own and which it will outsource and how that outsourcing will occur. This is an important discussion when talking about the starting a new product as a small business in the manufacturing space. Being efficient with operations requires maximizing the value produced from each activity for the corporation. For small companies this means making clear, overt decisions around what the company will do, and what the company will outsource. For smaller companies this can become a significant cost center, but is often still less expensive than providing the space, personnel and equipment for the product.

This paper examines a theoretical product for a theoretical company, the described operational approach and plan provided are hypothetical and utilized to explore particular operational management concepts. The product under consideration will have wide-spread appeal currently due to the nature of work-from-home arrangements for employers during the pandemic. However, its appeal my wane as working conditions change. Therefore a very important consideration for operations managers is time-to-market for this product.  Time-to-market has a significant correlation to firm performance and must be kept to the forefront of managers minds while rolling out this product \parencite{keawkuntiRelationshipSupplyChain2020}. This is not to say that product will have no perceived value for customers post-pandemic. Rather, it is to note that the product's primary appeal is going to be for those who are forced to work from home in a living situation where a private office in a segregated area of the home does not exist.

\section{The Product and Its Features}
\label{sec:org1040828}

\subsection{The Initial Product}
\label{sec:orga8df23d}

The initial product to be produced is a standing desk with a disappearing monitor / laptop compartment. Standing desks are very popular and provide a number of health benefits.  Further, there is evidence that standing desks increase attentiveness, lower stress, improve comfort, and may increase performance for some individuals \parencite{frostPatternAttentionStress2020}. These benefits have direct impacts on employers cost structures, as less healthy employees can increase benefits costs, lost opportunity costs, and retention rates. Therefore, while this product will be fairly costly, and perhaps even cost prohibitive for some people, corporations with work-from-home workforces are often times providing budget for necessary work accommodations for employees. It is not unrealistic or unheard of for companies to include proper desks and seating for employees. Therefore, the costliness of the manufacturing process will be offset by the fact that some of the costs will likely be absorbed by employer-paid expenses.

Further, many of those who would make use of this type of product will be in technology or leadership positions. This demographic will likely have some discretionary spending capability above the median, and therefore the impact of cost on the target market will be slightly diminished.

The desk will provide a work surface sufficient for two 24`` monitors or one 48'' monitor. The desk also have a dedicated stand to the left or right (customizable by the customer) that will hold up to a 17`` laptop. The desk will be pre-wired with power, usb-c, and display port connectors. The desk will have a servo motor that will be mechanically connected to two separate mechanisms that will be controllable by a small front panel control panel. One mechanisms will allow the desk to be raised and lowered to the desired height of the user, with the maximum height of the desk being 50'' and the lowest being 24``. The second mechanism will lower the surface components into the desk securing them away from view much like old fashioned sewing machine cabinets. This will present the user with an easy way to have a clutter-free table surface when not working. Of course, means for securing the keyboard, mouse, mousepad, and so forth will have to be designed.

This product has a market need at this time. More and more companies require their employees to work from home. This work-from home arrangement can create stress and reduce productivity for employees. This is in part due to not having adequate work areas available. It is also caused by the inability to differentiate between being ``at work'' and being ``at home.'' This segregation is easier for people who already have dedicated working spaces that are separated from the rest of the home. Thus, there is a need to segregate roles of being working or not in the home for those without dedicated office space. This product would provide for a clean boundary of which role a person is engaged in by clearly delineating when the work space is present and when it is not, which research has shown could result in lower voluntary turnover \parencite{rubensteinWorkhomeHomeworkConflict2020}. Further, providing a way to meaningfully manage clutter can help mediate overall happiness and life satisfaction \parencite{daoNegativeSideOffice2020} .

\section{Manufacturing Stages}
\label{sec:org47cceba}

There are multiple component parts that are necessary for this product. The simplest breakdown includes: the leg riser assemblies; the servo and mechanical assemblies; the wiring harness and connections; the collapsing top and hinge set; the stable top and hinge set; and any drawers or ornamentation elements. For a small manufacturer, one obvious way of tackling this problem with limited manufacturing and tooling is to produce each component in a separate batch, then do final assembly. This has several advantages:

\begin{enumerate}
\item It is cheaper to produce a whole batch than a single item due to having to reconfigure production machinery a minimum number of times.
\item This method helps with contingency planning, particularly in early stages of production as minor design changes can happen even during a batch process.
\item Planned setup time in between batches allows for maintenance and troubleshooting to happen on a pre-determined schedule.
\end{enumerate}

However, it also presents several disadvantages as well:

\begin{enumerate}
\item The need for space to store inventory as the component pieces are created.
\item Any quality errors could result in significant cost over-runs as entire batches need to be discarded.
\item Worker idle time during re-tooling periods.
\end{enumerate}

So, while not a panacea, it presents the most reasonable alternative for a small company that is going to do in-house manufacturing for a product of this complexity \parencite{schroederOperationsManagementContemporary2010} . However, that is only if all manufacturing is to be done in-house, which is not a necessary pre-condition for success.

There already exist small shops that would happily engage in small batch ordering as a supplier. Therefore, in order to minimize the need for real-estate and machinery investment up-front, the manufacturing process will be largely outsourced, with only final assembly being performed in-house.

Machining shops can easily provide the leg riser assemblies and hinge assemblies. Industrial electronics shops can provide the servo and mechanical assemblies and wiring harness and connections. Wood product companies can provide the desk surfaces, drawers, and ornamentation elements. Leaving only final assembly as the last step in the process necessary to create a final product.

However, even this is not necessary if the product design specifically calls for the customer to provide for final assembly. Then, what is necessary is that the company provide for designers, QA testers, and packing. The ``manufacturing'' process from the company perspective will include a small assembly line with a minimum number of workers which loads the components into the packaging, checks for completeness, and then seals and labels the packages. Finally, the company must package and ship the final units to retail customers.

Thus design, sales, QA, and final packaging/shipping will remain as in-house capabilities. All other manufacturing and assembly concerns will be outsourced in order to minimize initial costs. This is not a common method for starting a manufacturer in the era of ``Industry 4.0,'' this is a term that refers to manufacturing integrated systems that are defined primarily by being cyber-physical systems connected through the internet, and crossing organizational and corporate boundaries. It is a system that is defined primarily in terms of supply chain management first, and individual company internal operations management second. Industry 4.0 is named as it is seen as the fourth industrial revolution \parencite{sorooshianImpacts4thIndustrial2020}.

The first industrial revolution began in the mid-1700s with the invention of steam power. It slowly grew in scope with the increasing improvements of the machinery used to mechanize labor. The second industrial revolution arose in the late 1800s with the advent of mass production and electrical power sources. The utilization of conveyor belts and assembly lines created dedicated workers focused on specific tasks to increase the speed of production and lower costs. In the 1969, the introduction of printed circuit boards heralded in the third industrial revolution.  This allowed for the automation of some task and computer monitoring of in-house manufacturing capabilities. The fourth industrial revolution was so-named in 2011. It refers to a market where the exchange of information, made possible by the internet, which creates closely coupled and tightly integrated networks of manufacturers into a single system that is capable of being driven by data analytics, artificial intelligence systems, autonomous robots, and additive manufacturing (ibid).  This product is envisioned to be a child of Industry 4.0, so much of the considerations for operations management is rightly focused on supply chain management.

\section{Operations and Supply Chain}
\label{sec:orgcce89b1}

The decision to outsource all manufacturing components presents several small hurdles as a company. However, it comes with several proven advantages. For example, \textcite{bengtssonLowCostInnovationContrasting2009}  showed that ``Firms that apply low-cost-oriented outsourcing will lower the product development costs of out-sourced products.'' Further, if the company seeks out innovation-oriented outsourced suppliers, the firm can also significantly increase its own ability to innovate in design. This will help improve both time-to-market and functionality of the product.

This same study notes the important caveat that the suppliers must be involved early in the design process. Indeed, work by \parencite{fagerstromEfficientCollaborationMain2002} shows convincingly that the importance of supplier involvement increases with product complexity. As this is a moderately complex product, it is necessary to tightly integrate the company's design functions with the  supplier's operations engineers to ensure their capability to deliver as well as to ensure that the sub-assembly quality specifications are sufficient to ensure success.  This consideration is only one portion of supply-chain integration, which will be necessary to fully realize the financial and operational benefits of outsourcing for a firm that is critically short of internal resources \parencite{dekkerFirmEnablementOutsourcing2020}.

Because of the system complexity as well as the need to constrain costs, supply chain management for this company must include a significant investment in information technology considerations to integrate with the upstream suppliers. This will require top management support to understand that IT investment is not merely a ``nice to have'' area where costs can be cut, but rather a significant area of competitive advantage \parencite{daneshvarEffectiveFactorsImplementing2020}. Thus IT investment becomes an important operational consideration for supply chain management.

It is well understood that supply chain intelligence, allowing for just-in-time production produces significant cost savings. Thus, a significant aspect of the supply chain management side of operations management for this product will focus on selecting, grooming, and supporting suppliers in producing an integrated supply chain \parencite{yangAchievingJustTime2021}.   This is necessary to enable the short start-up, time-to-market advantage the company seeks in order to generate market interest during the pandemic \parencite{ghobakhlooCorporateSurvivalIndustry2020}.

\section{Operational Evaluation}
\label{sec:orgc17a42f}

Key areas of consideration for the operations management of the supply chain start with supplier selection. Once the SCM system(s) are decided upon, suppliers in the supply chain must be evaluated on two axis: first, the ability to deliver to specification; and two, the ability to fully integrate into the supply chain. Neither of these are optional conditions, but it must be understood that both should be considered in terms of future probability ranges. It is easy to lay claim to a certain standard of quality in manufacturing, but the demonstrable proof of that capability is delivering to the standard. Likewise, merely having an SCM system capable of integration can be a far cry from actually having the technical competence to achieve integration.

Thus multiple suppliers for components should be sourced initially. While this will raise the development cost of building the final integrated supply chain, it will lower long-term operational costs by raising the likelihood of the successful creation of a truly integrated supply chain. The first key-performance indicator will be the presence of multiple suppliers with minimum delivery capabilities that have been assessed and documented.

Once the suppliers are selected and integration into the SCM has started, test batches of components must arrive for evaluation for fitness. If the test batches pass QA checks at the final stage, then QA examinations of products must happen at the stages of manufacturing of the sub-components, and the results delivered to the integrated SCM system. This will allow operations management to understand the quality impacts and costs associated with each supplier. Adjustments to order volume, frequency, and ratios must occur regularly in order to maximize the quality of the delivered product to customers. This is necessary as this will be a relatively high-cost item in the market space, and at the price point envisioned, quality of the final product will be a major factor in customer satisfaction \parencite{yoonSystematicApproachPrioritizing2020}.

This product as envisioned will require final assembly by the customer. Thus R\&D expenditures must include time focused on instructional materials, packaging, and post-sale customer service. It may be necessary to address shipping issues promptly, and how to integrate post-sale customer demand for parts into he shipping process may be a deciding factor between being perceived as a solid product sold by an outstanding company and being thought of poorly by customers. For example, if a servo component is found to not work properly by the customer, it will be costly to drop-ship a new one overnight. These are relatively bulky and heavy components. However, failing to deliver this component to the customer quickly may result in a return of the item, a lost sale, and potentially negative social-media attention.  This would not be a desirable outcome. Thus, internal process integration will be necessary to be fully successful.

Still, with an integrated supply chain, it is not essential that the part be shipped from the assembly plant to the customer. It can, and likely should, be shipped from the contracted supplier. They, after all, as the source of the components will be the ones that have them readily available. Therefore, an important enhancement to the customer service experience will be to ensure that the customer management system integrates into the supply chain management tool set in such a way as to allow customer service representatives to place orders for drop shipping of components to end-customers and intermediaries from the suppliers. This will call for not only additional IT integration efforts, both internally and externally, but also providing branded shipping materials, instructions, and other packaging to the suppliers in the supply chain.

This level of integration is unlikely to happen immediately due the level of IT work involved. But it should be considered at the time of contracting as a desired future state, both so that both companies can consider the involved costs in the bidding process, and so that all of the involved parties are more-or-less bought into the idea for future IT work aimed at further integrations of the supply chain.

\section{Comparing Products and Plans}
\label{sec:org38c6244}

There several products that are comparable to one feature, that of the standing desk. Ikea for example, sells the several models of standing desks, including those with automated servo system for raising and lower the desk via a front-panel control system. However, the Ikea products do not have any comparable system for hiding away the computer components on the top of the desk. Ikea's supply chain is well-known for its tightly integrated nature. Ikea's supply chain is actual structured to be ``circular,'' with the end-of-life of the product taken into account for how it might be recycled in future manufacturing steps. Ikea has noted that the global pandemic has impact some sales, but sales of office furniture has actually increased \parencite{ScenesStrategicIkea}. Ikea does some of their own manufacturing, but they also engage in outsourcing of products or product components. This suggests the model suggested for this product is not unrealistic compared to existing areas of the industry.

Another product that is comparable in features are sewing cabinets such as the Arrow Homespun Bertha Sewing Cabinet. This product has entirely manual operations, but is designed to hide clutter at the end of use, including the lowering of the sewing machine into the cabinet body. This product is likewise designed to be assembled by the end customer. Reviews from the website indicate that customers love a well-built, sturdy piece of furniture that hides away clutter but expands to provide a large work surface. Customers are not put off by building it themselves \parencite{joann.comArrowHomespunBertha}. Indeed, there is some evidence from the field of behavior economics that customers perceive the realized value of items they assemble as slightly higher than items that they merely purchase.

There are no competing products in the market place that provide both features simultaneously. However, that is not of any grave particular concern, as this product is imagined merely as the integration of existing technologies (hinged table surfaces and telescoping table legs) into a single product. The real engineering challenge will be ensuring the safety of the monitor(s) during operation. However, the existing products that offer similar functionality and features suggests that this product can be successfully produced. Moreover, the customer-assembly nature of individual components strongly suggests that realizing this product with an integrated supply chain approach is realistic and achievable.

Of some concern is the recognition that both Ikea and Arrow design furniture with a fairly lengthy time-to-market. There is real effort required to ensure the manufacturing tolerances for each component are consistently realized in order to ensure the final product will satisfy customers. Customer assembled furniture suffered from terrible perception problems after initial release, being seen as low quality and cheaply made. This perception was well earned by manufacturers that were seeking the lowest possible cost with the minimal accepted quality standards. In order to avoid being perceived in this light, the company will need to focus on design and engineering tasks, work with integrated suppliers early to ensure quality metrics are in place, and closely monitor the supply chain for defects. Further, customer relations management will need to integrated into the supply chain to ensure quick and painless drop-shipping of damaged or missing parts. This is again an area where Ikea shows it is possible to perform to customer's expectations.

One serious consideration should be if the company can not avoid supply chain management altogether and simply design products and license the resulting design to an existing large, integrated manufacture such as Ikea or Arrow.  While this idea is beyond the scope of the paper, it should be noted that most important component of Industry 4.0 is information, not products. It is information sharing and flow that allows for Industry 4.0 companies to thrive, and if the product can be realized and provide for a viable revenue stream by ``outsourcing'' all of the manufacturing, shipping, and customer service components through licensing structures, then that possibility should not be overlooked.

\printbibliography
\end{document}
