\section{Challenges for Business Process Improvement}
\begin{frame}
  \frametitle{Challenges for Business Process Improvement}
  \framesubtitle{Barriers to eliminate waste}
  \begin{itemize}
    \item <1-> Automation
      \note[item] <1-> {\scriptsize{For the automation option, the barrier is increased business risk more than increased costs. Because the company is in a highly regulated industry, their internal IT practices are subject to external audit. If the company chooses to automate security patching, they will be fully liable if their automation tools fail and the company is actually exposed to a security breach. By keeping the process manual, though expensive, the company's liability can potentially be reduced as they company can point to the vendor as being complicit in any failure due to the lack of automation on the vendor's part \parencite{hahnLawEconomicsSoftware2006}. While this is the most cost effective recommendation, it is the least palatable to management for this reason.}}

    \item <2-> IaaS
      \note[item] <2-> {\scriptsize{Moving infrastructure to the cloud likewise has challenges. Management is concerned that giving up complete control of the underlying infrastructure is itself risky. This conception however is flawed \parencite{zhangWeKnewIt2014}. The IaaS model can be expensive to implement as it will usually require a significant business transformation. But the result is actually the total elimination of this process for most devices!}}

    \item <3-> Indetification
      \note[item] <3-> {\scriptsize{Since not all Infrastructure will be able to be moved to the cloud without a larger technology renovation project, the key to success will be to identify the systems and devices that can be most easily moved, which have the highest associated patching cost, and which are least susceptable to automation. }}
\end{itemize}
\end{frame}
