\section{Electronic Security - Patch Management}
\begin{frame}
  \frametitle{Introduction}
  \framesubtitle{Understanding Patch Management for Complex IT Environments}
  \begin{itemize}
    \item<1-> The security components of modern IT systems are sprawling
      \note[item]<1-> {\scriptsize{Modern IT security relies on a concept of ``defense in depth,'' which involves multiple components working in concert to protect the environment. These devices include firewalls, load balancers, intrusion detection systems, endpoint defenses, network devices such as routers and switches, and user security measures \parencite{cisdid}.}}

    \item<2-> Ensuring devices are patched is a critical component of security
      \note[item] <2-> {\scriptsize{Keeping devices in the network patched is critical for both operational efficiency (patches that improve device function) and for maintaining security (patches that address vulnerabilities to the system.) It is critical for organizations to understand which patches are which for the devices under management in their system. Security patches must be applied promptly while operational efficiency patches may be applied only as needs of the organization dictate \parencite {dhs2016}.}}

    \item<3-> Patch management is complex and costly
      \note[item] <3-> {\scriptsize{There is no ``one-sized fits all'' solution for security patch management and most organizations find that patch management is a costly endeavor. Patching involves human labor, risk of outages, system disruption, and possible operational changes \parencite{dey2015}. Moreover, most organizations have multiple vendors to consider as their device suppliers.}}

  \end{itemize}
\end{frame}
