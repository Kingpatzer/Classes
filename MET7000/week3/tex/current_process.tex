\section{Current Process}

It can be surprising to discover that the manufacturing process of Jelly Belly jelly beans can take as many as 14 days to complete \parencite{foodinsiderHowJellyBelly2019a}. The process is a complex combination of manual and automated steps, and the timeing is dependent upon the food chemistry involved. The basic outline of the process, according to Food Insider, is:


\begin{enumerate}
  \item A slurry of water, corn starch, sugar, and corn syrup is made for the batch
  \item Once the slurry is ready, flavorings are added. These are a specific mixture of juice concentrates, fruit purees, and manufactured ``natural'' flavorings.
  \item The resulting flavored slurry is fed into a machine known as a ``Mogal'' which creates a candy mold out of corn starch, and injects the flavored slurry into the mold.
  \item The candy is then left in the dry corn starch molds in a tempurature and moisture controlled environment to firm up.
  \item The candy is then turned upside down and placed back into ``Mogal'' machine where the jelly beans are given a steam bath and passed through a sugar shower. This prevents the candy from sticking together during the next phase of of the process.
  \item Next, colored syrups and sugars are and poured on the candy is it goes through a rotating drum system. This creates the first layer of the candy shell.
  \item The jelly beans are again set aside to rest and setup
  \item The jelly beans are put back into the rotating drum system to receive a second coating of syrup for the shell
  \item Now, the jelly beans are topped wtih confectioner's glaze and bee's wax
  \item At this point, the batch is passed through a quality control check where the beans are inspected for shape and consisitency, as well as being taste tested.
  \item Having passed the QA checks, the batch will then be sent to a printing machine where they recevie the jelly belly logo stamp.
  \item Finally, a machine packages the Jelly Belly into bags, which are then packed by hand into boxes for shipping.
\end{enumerate}

In measuring a process flow, we must observe \textit{Little's Law}. This mathematical relationship between inputs and outputs states that the average number of items in the process flow is the product of the arrival rate and the time that any item stays within the process. This law seems amazingly simply, but has been demonstrated to govern operations processes in a varity of contexts. The importance of this Law in this context is straightforward, the operations manager can only really control two of the three variables, the third will be determined by the other two \parencite{littleFORUMLittleLaw2011}. Therefore, in analyzing the Jelly Belly process for issues or inefficiencies, we must recognize the limitations that the process chemistry provides. Given that depending on the flavor, the process is time bound by the food chemistry that must occur to between 7 and 14 day, which means that the only variables the operations manager can shift is either the arrival rate of new items into the system or the average number of items in production at any one time.

In the real world, and certainly within Jelly Belly, both the arrival rate of new items into the system adn the average number of items in production at any one time are constrained by physical factors. For example, the rooms where jelly beans are dried and setup are a specific limited size, and have a maximum number of items which can be stored within them at any one time. Likewise, storage space for raw materials, as well as the operational speed of the docks which bring raw materials into the system face physical constraints due their size and design.

With Little's Law in mind, let's move on to an anlysis of the process itself.
