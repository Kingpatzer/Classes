\section{Conclussion}

The operation of Jelly Belly is a complex interaction of multiple batch processes comprising of a significant number of individual steps. There are several places where the process flow slows creating constraints to throughput. Additionally, there are multiple independent human labor components that present opportunities for cost reduction through additional automation.

Jelly Belly's products include setup processes which are extremely expensive in terms of time, and require significant storage space for the holding of inventory in this state. There are advances in food chemistry which may be brought to bear for Jelly Belly to reduce the time and space requirements of these steps. Jelly Belly would need to explore these possibilities with appropirate R\&D projects.

Jelly Belly additionally has several points where the production process consumes material that does not become part of the final product. Again, these areas should be explored for re-tooling and adjustment to save the costs associated with having to ship, store, and consume materials that do not become part of the final product itself.
