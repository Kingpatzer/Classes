\section{Strategy Formulation}

In formulating the strategy to be considered, Accenture's mission, values, and vision were assessed along the eight axes Tracy (2015) provides for strategic variables, and the four basic strategies: specialization, differentiation, segmentation, and concentration. With respect to these considerations, market segmentation was the area of most strategic concern.

The reason for this concern rests on the fact that so many rising companies, with billion-dollar valuations, are only a few years old. Whereas in the past, it would take decades to become on of the most valued companies in the world, today it takes far less time. Indeed, today, the average S\&P Fortune 500 lifespan is down to 15 years. Further, more than 75\% of UK Fortune 100 companies have dissappeared in the last few decades \parencite{hillHowWinningOrganizations2018}. Given that Accenture's top clients have relationships that last more than a decade, this demonstrates both a risk and a missed opportunity.

With respect to Specialization, similarly to Segmentation, Accenture's strategy needs to shift to determine how to engage smaller, developing companies that have bright futures and significant technology challenges.

Strategically speaking, Accenture's current differentiation and concentration directions aligns well with Accenture's mission, vision and values. From the standpoint of differentiation, Accenture should continue course, or accelerate investment into R\&D for cloud technologies, and other technology spaces where Accenture remains a standout.

The primary concern for the formulation of a go-forward startegy is how to operationalize engaging high-value, small comapanies with high growth potential.
