\section{Recommendation}

Having analyzed how the various strategic variables identified by Tracy align to Accenture's mission, vision, and values, it is time to consider which specific strategy to recommend.

\subsection{Specialization}

Accenture's current specialization is on providing clients with technology strategy, delivery, and operational life-cycle support. Accenture is not specialized with regard to geographic, horizontal, or vertical markets. Accenture is specialized to handle large, established enterprises who can engage Accenture as a long-term strategic partner.

The specialization strategy for Accenture should consider changing with respect to customer selection, as the rise of cloud computing allows for small, relatively new companies to become important global companies nearly overnight. Developing a reliable path to developing relationship with these potential future customers early is likely necessary as market changes mean that today's established enterprises aren't likely to be tomorrow's.

This focus is not unique among Accenture's competitors, but the number of competitors who can provide this specialization at scale is limited, and will remain so. Another reason for determining how to engage smaller companies on the rise is that regional consulting firms, an alternative to global consultancies, can ride relationships with growing smaller companies into the global space.

\subsection{Differentiation}

Accenture differentiates itself from its competitors across nearly every strategic variable. Being the largest, most globally diverse consulting company simply makes Accenture stand out from all competition. In competitor analysis, Accenture was found by Gardner Group to stand out in multiple arenas. This will not change in the near future, and Accenture can utilize its financial strength to continue to excel both by expanding capabilities and continuing to invest in R\&D, training, and innovation.

Accenture's competitive advantages are essentially three: being able to provide any technology strategy, delivery, or support service anywhere in the world; being a thought leader with respect to developing technologies and innovative applications of emerging technologies; and, having global expertise in literally every industry to allow for cross-pollination of ideas to provide best-in-class services to all customers.

This strategy's downside is that its is costly to implement and sustain and therefore Accenture can not compete on cost in the market place. Accenture's focus on, and reputation for, long-term value creation as a strategic partner justifies their premium price, however, this competitive advantage also limits Accenture's potential customer base.

This strategy essentially has defined Accenture up until this point, and should not change, as it is the area where Accenture is a global leader with literally no peer in the market.

\subsection{Segmentation}

Accenture's market segment is larger corporations who wish to engage in long-term relationships with Accenture. As noted previously, most of Accenture's best clients are also Accenture's oldest clients, many with relationships that go back decades.

This strategy should change.

It is well established that some of the largest companies in the world today simply didn't even exist 10 years ago. DoorDash, Instacart, Robinhood, SoFi, Impossible Foods, Casper, Lyft, ZipRecruiter, Instagram, Gusto, Snapchat, Coinbase and many others are all examples of multi-billion dollar companies that are less than a decade old. Some rose from nothing to multi-billion dollar enterprises in only a few years.

Accenture needs to develop a segmentation strategy that continues to value long-lasting relationships with clients, but which actively seeks out small companies with global potential.  Accenture could perhaps use some of its financial might to engage in venture capital activities, and then leverage its strategy and development capacity to assist the ventures which it invests in. This would position Accenture to be a partner of choice long-term for new companies that are quickly entering the global cloud-based market space.

\subsection{Concentration}

Accenture's strategic concentration on technology strategy, delivery, and operations should not change. In this day and age, all companies need technology to compete, but it is rarely the core competency for any one company. Further developing systems that bridge vertical, horizontal, or global markets to drive efficiencies is a competency that few companies posses. This focus gives Accenture a competitive edge over all but a select few competitors.
