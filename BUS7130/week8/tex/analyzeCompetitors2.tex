\begin{frame}
  \frametitle{Analyzing Alternatives Competitor}
  \framesubtitle{The Secondary Market}

\begin{itemize}
  \item{Regional Firms}
  \item{Boutique Firms}
  \item{DIY}
\end{itemize}


\note[item]{\scriptsize{Regional firms can have real advantages for clients. Chief among these is a deep understanding of local customs, practices, laws, and norms. This advantage extends over global firms such as Accenture due in large part to global firms generally being constrained by global norms and internatioanl laws in a way that local regional firms may not be \parencite{chuanhengliTrajectoryConsultingFirms2005}. }}
\note[item]{\scriptsize{Boutique firms are narrowly specialized, often in a single technology stack or implementation methodology. Their tigth focus allows them to develop a high degree of expertise and a strong reputation relative to their size. Clients with well-defined project needs will often seek out a boutique firm. Like regional firms they are often highly price competitive as well. }}
\note[item]{\scriptsize{In-House capabilities are often attractive to leadership. It is assumed that by developing in-house capabilities across IT can lead to competitive advantage for a firm. This is rarely the case, as a lack of specific knowlege and capabilties rarely provides the ROI necessary to justify the costs to the firm \parencite{dossantosJustifyingInvestmentsNew1991}. Still this option does attrack many companies and Accenture needs to carefully articulate the value proposition of utilizing Accenture over in-house capabilities for any functions that are not at the core-competency level of the client.}}
\end{frame}
