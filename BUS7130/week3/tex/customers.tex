\section{Key Customer Data}

There are several obvious commonalities across the most important of Accenture's clients. First is size and scale. Accenture currently is serving over 6,000 clients globally. But these are not small entities. Clients are other large scale organizations that can make investments in technologyf frmo \$10s of millions up to multiple hundreds of millions per quarter. The Diamond Clients, in 2019, consisted of almost entirely Global Fortune 1000 companies. These included 91 Fortune Global 100 clients.

Another key aspect of Accenture's client base is the length and depth of the relationships developed. 99 of the top 100 revenue clients in 2019 where clients for at least 5 years; and 95 were clients for at least 10 years. This trend extends beyond the top revenue generating clients. Accenture's clients tend to develop long-term relationships with Accenture in part because Accenture overtly seeks to become an essential strategic partner with the client; and in part because Accenture helps the clients identify and tackle technology projects that are multi-year in scale and scope.

Given size of engagements that Accenture seeks, clients are limited to large entities with extensive revenues and government entities with large budgets. The average reported revenues of Accenture's publically traded clients exceeds \$500M annually, and the vast majority exceed \$1B.  Clients are generally national or global in their own scope, though larger regional entities, particularly insurance and financial service providers, do make the client list. The five operational groups break down to 13 focused industry groups, and more than 40 industries. Accenture does not make the data at these levels of breakdown public.

Most of Accenture's largest Diamond+ clients utilize Accenture across 2 or more of their delivery channels: Strategy and Consulting; Interactive; Technology; and Operations. A typical engagement could include Accenture Technology implementing new features, products and services while the operation of those items are handled by Accenture Operations. It is rare for Accenture Strategy and Consulting to be engaged without any of the other teams also being engaged. When there is only a single delivery channel engaged, it is typically Accenture Technology, who is responsible for the design and implementation of technology solutions.

While Accenture is the largest global consultancy by far; it is not uncommon for clients to include Accenture competitors in their mix of suppliers. Accentures' clients have a large number of consultancy companies to select from it is not uncommon for them to spread work around to multiple large companies. The most common competitors to find also engaged at Diamnod level clients are Capgemini, Infosys, Cognizant, KPMG, PwC, Earnst \& Young, and IBM.  Accenture views the presence of these competitors within a client as a net benefit. Accenture can gain insight into how Accenture's competitors are operating; and Accenture can demonstrate to the client a willingness to engage competitors with full levels of collaboration to the client's benefit.

Accenture's clients typically have very high quality standards. Clients tend to be highly successful companies in industries that are frequently highly regulated. They have little ability to tolerate lapses in quality, particularly with operationally critical technology. Quality and speed of delivery are the two most important factors Accenture clients cite when asked why they choose Accenture or remain and Accenture client.

Accenture is well-regarded as a primary implementor of customer relationship management (CRM), enterprise resource planning (ERP), enterprise resource mamanagement (ERM), and similar enterprise scale systems. These systems are highly complex, and require significant expertise to congfigure correctly \parencite{aboabdoImplementingEnterpriseResource2019, farhanSystematicReviewDetermination2018}. This is expertise that the client is unlikely to posses directly. Moreover, often the client will need to engage in cultural transformations to be successful. Such transformational needs often require a trusted partner for success \parencite{kimpelMeasuringImpactCulture2020}.

Accenture does face challenges in meeting client expectations with regard to localized demands. While Accenture's scale and long-standing client relatioships make it very easy to know the most important clients well, Accenture's growth prospects are focused on ``Growth Market'' companies \parencite{AccenturePLC2019}. However, Accenture's scale and focus on the needs of Diamond tier clients first can make adjusting to fit the client needs and expectations in these growth markets difficult \parencite{tinneStrategicIssuesAccenture2016}

Accenture's customers are facing two critical technology challenges. First is the ability to be more ``agile'' with respect to technology delivery. The second is cost containment \parencite{bohmannServiceAgilityAssessment2005}. Accenture has identified cloud computing as an ideal solution for both of these issues for clients of any size or scale. Only approximately 20\% of Accenture's current client base are utilizing the cloud, and almost none are fully leveraging the cost saving opportunities presented by cloud technology. Accenture believes that cloud technology is a true revolution in computing that will have the same sort of impact as moving from mainframe to desktop computing, or moving to the web have had in the past. Of key consideration is determining how to help customers adopt cloud computing \parencite{vuPredictorsCloudComputing2020}.
