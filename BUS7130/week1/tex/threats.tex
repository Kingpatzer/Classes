\subsection{Threats}

A number of the largest growing competitors to Accenture, such as Cognizant, are primarily staffed out of India and Asia. This allows them to present themselves as low-cost alternatives in the marketplace. While none of these companies have the global reach of Accenture, the competition for services in countries where Accenture and competitors co-exist has become much more cost-competitive. While Accenture continues to dominate these large-scale, low-cost alternatives, they are making headway into accounts that Accenture previously thought of as 'safe,' making Accenture continually re-evaluate their value proposition for clients.

As these companies are growing, there is a market for top-tier consulting talent that is tightening. While Accenture is in the process of reducing staff, which may likely result in Accenture being more innovative in the near term \parencite{ritter-hayashiSuccessBelongsFlexible2020}, Accenture will continue to face the difficult task of finding and retaining the highest possible quality staff.

Finally, Accenture must contend with growing global political instability which itself contributes to economic uncertainty \parencite{siuChinaBeltRoad2019}. There is little doubt that as a global company, Accenture must be continually aware of, and respond to, the changing political dynamics of international relations between countries. Numerous Accenture employees work in countries other than their own, and growing travel restrictions and trade embargos can only decrease Accenture's ability to fully utilize its staff in the most economicly beneficial way possible.
