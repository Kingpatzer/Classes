\subsection{Weaknesses}

As stated previously, Accenture's bench is relatively inactive. Accenture aims for 5\% staffing overcapacity, but is currently at roughly 10\%. A certain amount of ``excess'' overcapacity is expected simply due to having to maintain corporate functions, allow for training, vacation and so forth. So, in order to have 5\% of staff available for re-assignment, typically 7-8\% of staff needs to be in a non-billing status at any one time. This has led to Accenture announcing a reduction of staffing of roughly 5\% \parencite{owusuAccentureCut25}.

Additionally, Accenture has assiduously avoided creating secondary products or looking for secondary markets for products that Accenture develops. It is the corporate philosophy that Accenture serves it's clients best by not having products of it's own to offer. Rather, Accenture wants to stay a neutral partner with respect to any particular technology stack. However, this means that Accenture does not turn any of it's R\&D offerings into products from which it can derive income from sales or licensing fees. This limits Accenture's overall earning potential during market downturns. Typically firms which are committed to particular software and/or technology stacks must utilize their tools even during periods of market volatility. This means that licensing fees are relatively stable sources of income even in down markets. Accenture does not presently avail itself to this source of revenue as a strategic choice. However, this choice weakens Accenture's financial stability in a down market.
